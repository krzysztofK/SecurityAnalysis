\chapter{Analiza zastosowań tworzonego systemu}
\label{cha:zastosowania}

{\it

Niniejszy rozdział przedstawia możliwości zastosowania proponowanych rozwiązań w rzeczywistych systemach o architekturze SOA oraz wynikające z tego korzyści. Wykorzystanie mechanizmów systemów zarządzania tożsamościami pozwala zapewnić bezpieczeństwo aplikacji o architekturze zorientowanej na usługi. Zastosowany standard SAML i definiowane przy jego użyciu mechanizmy stanowią rozwiązanie dobrze dopasowane do specyfiki systemów o architekturze SOA. Elastyczność i skalowalność proponowanego podejścia sprawiają, że możliwe jest jego efektywne wdrożenie w rzeczywistym środowisku systemu bazującego na architekturze SOA.

Druga część rozdziału podsumowuje przydatność głównych rozwiązań technologicznych zastosowanych w projekcie w tworzeniu systemów o charakterze SOA.

Rozdział opisuje również potencjalne ulepszenia proponowanego rozwiązania. Zwiększenie poziomu bezpieczeństwa i rozszerzenie funkcjonalności systemu mogłoby być możliwe np. dzięki: wprowadzeniu wielu usług uwierzytelniających STS powiązanych relacją zaufania, zastosowaniu szyfrowania i podpisów na poziomie fragmentów przesyłanych wiadomości oraz możliwości definiowania parametrów bezpieczeństwa przy użyciu polityk.

}

%---------------------------------------------------------------------------

\section{Możliwości wykorzystania środowiska w praktyce}
\label{sec:wykorzystanieWPraktyce}

	System powstały w wyniku wykonanych prac projektowych stanowi przykład wykorzystania podejścia opartego o zarządzanie tożsamościami jako metody zapewnienia bezpieczeństwa dostępu do aplikacji w architekturze SOA. Przygotowana implementacja dowodzi, że  istnieje możliwość oparcia funkcjonowania mechanizmów bezpieczeństwa aplikacji w architekturze SOA na rozwiązaniach dostarczanych przez systemy zarządzania tożsamościami. 

	Wynikiem prac implementacyjnych jest system wykorzystujący standard SAML jako realizację założeń systemów zarządzania tożsamościami. Przygotowany system dostarcza przykładów zastosowania specyfikacji SAML jako metody uwierzytelniania klientów usług sieciowych. Wykorzystanie  SAML w procesie uwierzytelniania umożliwia implementację mechanizmów bezpieczeństwa dla usług opartych o różne standardy dostarczania usług sieciowych(np. SOAP i REST). Do architektury przykładowego systemu wprowadzono również moduł magistrali usług - ESB, będący charakterystycznym elementem systemów opartych o architekturę SOA. Zaproponowana implementacja modułu magistrali usług dostarcza funkcjonalności przetwarzania wiadomości z dołączonymi tokenami bezpieczeństwa - asercjami języka SAML. Dzięki zastosowaniu mechanizmów bezpieczeństwa bazujących na specyfikacji SAML możliwe było osiągnięcie funkcjonalności jednokrotnego uwierzytelniania klienta dla różnych usług systemu o architekturze typu SOA. Mechanizm jednokrotnego uwierzytelniania może być wykorzystany podczas budowania procesu biznesowego odwołującego się do różnych usług sieciowych.

	Zastosowanie mechanizmów systemów zarządzania tożsamościami w kontekście aplikacji w architekturze SOA jest rozwiązaniem posiadającym wiele zalet. Może stanowić sposób standaryzacji implementacji mechanizmów zapewniania bezpieczeństwa aplikacji. Opiera się na oddelegowaniu odpowiedzialności związanych z bezpieczeństwem aplikacji do specjalizowanych usług. Dzięki zastosowaniu standardu SAML możliwe jest tworzenie relacji zaufania pomiędzy różnymi usługami uwierzytelniania i autoryzacji. Dzięki temu klient aplikacji może w transparentny dla siebie sposób korzystać z usług w obrębie różnych domen bezpieczeństwa. Zastosowanie zarządzania tożsamościami jest podejściem elastycznym, skalowalnym, umożliwiającym efektywne rozszerzanie na kolejne elementy infrastruktury systemu. Oddelegowanie odpowiedzialności związanych z uwierzytelnianiem i autoryzacją do odrębnej, specjalizowanej usługi zwiększa poziom bezpieczeństwa systemu. Wykorzystanie specyfikacji SAML umożliwia realizację procesu jednokrotnego uwierzytelniania klienta korzystającego z wielu usług sieciowych.

	Proces zapewnienia bezpieczeństwa dostępu do systemu opartego o architekturę SOA może być realizowany z wykorzystaniem mechanizmów zarządzania tożsamościami. Zaproponowane rozwiązanie wykorzystujące standard SAML jest podejściem dobrze dopasowanym do specyfiki i wymagań systemów o architekturze SOA.

%---------------------------------------------------------------------------

\section{Użyteczność zastosowanych technologii w systemach udostępniania usług}
\label{sec:uzytecznosc}

Przy pomocy wymienionych w tekście pracy technologii, udało się stworzyć prosty system obsługi zleceń w sklepie internetowym. Sam ten fakt dowodzi już, że wybrane technologie można z powodzeniem zastosować w aplikacjach o charakterze SOA. Jednak poszczególne elementy budujące system mogą się od siebie znacznie różnić pod względem użyteczności.

Java jest prawdopodobnie najpopularniejszym językiem służącym do tworzenia systemów zorientowanych na usługi. Pewne cechy języka zapewniały mu pozycję lidera w klasie aplikacji typu enterprise jeszcze przed spopularyzowaniem konceptu SOA. Sam język jest stosunkowo łatwy w nauce i popularny. Zapewnia wsteczną kompatybilność, bardzo ważną dla systemów korporacyjnych, które często cechuje długi cykl życia. Posiada bardzo dobrą dokumentację i rozbudowaną bibliotekę standardową, umożliwiającą przetwarzanie wiadomości w formacie XML(JAXP, JAXB). Java Enterprise Edition definiuje dodatkowo wiele API szczególnie przydatnych w aplikacjach SOA, takich jak obsługa usług sieciowych oparta o SOAP(JAX-WS) i REST(JAX-RS), współpraca z systemami kolejkowymi(JMS), persystencja danych(JPA), zarządzanie transakcjami(JTA), komponenty EJB, czy w końcu deklaratywne zarządzanie bezpieczeństwem. Platforma uruchomieniowa Javy zapewnia przenośność napisanego w tym języku oprogramowania, co jest niezwykle istotne w heterogenicznym środowisku w jakim często uruchomione są systemy SOA. Nie bez znaczenia jest też automatyczne zarządzanie pamięcią(garbage collection) i bezpieczeństwo niskopoziomowych operacji zapewniane przez środowisko uruchomieniowe, eliminujące możliwość niektórych rodzajów ataków, np. przepełnienia bufora. 

Potencjalna użyteczność Javy w tworzeniu oprogramowania typu SOA zaowocowała duża ilością bibliotek, frameworków i innych narzędzi ułatwiających tworzenie aplikacji tego typu w tym języku. Co za tym idzie, organizacje które zdecydowały się tworzyć system w oparciu o język Java mają możliwość wyboru rozwiązań dopasowanych do ich potrzeb oraz ewentualnej migracji na inne rozwiązania w przyszłości. Popularność tej platformy skutkuje również tym, że znaczna część publikacji dotyczących architektury zorientowanej na usługi przyjmuje Javę jako wzorcowy język implementacji.

JBoss jest serwerem aplikacji zgodnym ze specyfikacją serwera Java Enterprise. Dostarcza on implementacji API definiowanych przez standard Java EE oraz własnych, bardzo użytecznych rozszerzeń, np. gotowych komponentów uwierzytelniających użytkownika na podstawie serwera LDAP lub bazy danych. JBoss dostarcza także typowych funkcji dostępnych również w konkurencyjnych serwerach takich jak monitorowanie i zarządzanie aplikacją z poziomu interfejsu webowego. Wybór JBossa jako kontenera aplikacji pociągnął za sobą kilka kolejnych wyborów technologicznych, gdyż można było się spodziewać, że biblioteki stworzone przez organizację Red Hat, zajmującą się rozwojem serwera, będą z nim dobrze współpracować.

Zgodnie z przewidywaniami, framework Picketlink dobrze integruje się z serwerem JBoss. Dostarcza on gotowych, rozszerzalnych komponentów zarządzających uwierzytelnianiem i autoryzacją użytkowników: Security Token Service dla serwisów sieciowych i Identity Provider dla aplikacji webowych. Projekt ten intensywnie się rozwija, sukcesywnie dodawane są do niego nowe funkcje i obsługa kolejnych standardów związanych z bezpieczeństwem. Pewnym problemem była dosyć uboga i fragmentaryczna dokumentacja, jednak w ostatnim czasie została ona znacznie poszerzona. Dodatkowo, autorzy biblioteki nie gwarantują poprawności działania z innymi serwerami aplikacyjnymi.

Możliwości jBPM nie zostały do końca wykorzystane w projekcie. Narzędzie to służyło jedynie do prostej orkiestracji usług i z powodzeniem spełniało tę funkcję. Jako zaletę można uznać fakt, że jest to prawdopodobnie najlepiej udokumentowany projekt organizacji RedHat który został użyty w tej pracy.

Przydatność SwitchYard w kontekście aplikacji SOA jest trudna do przecenienia. Framework posiada wszystkie zalety rozwiązań typu ESB. Zapewnia transparentność lokalizacji usługi, standaryzuje transformację wiadomości i zamianę protokołów, a w połączeniu z projektami Camel, jBPM i Drools umożliwia także dynamiczny routing i rozszerzanie wiadomości. Umożliwia także weryfikację kontekstów bezpieczeństwa i transakcyjności. Dostarcza ponadto możliwości zarządzania i monitoringu udostępnianych serwisów poprzez rozszerzenie panelu administracyjnego serwera JBoss.

OpenShift umożliwia łatwe wdrożenie aplikacji w środowisku chmury obliczeniowej. Platforma ta dobrze współpracuje z pozostałymi wymienionymi technologiami, dostarczając „out of the box” wsparcie m.in. dla serwera JBoss oraz frameworku SwitchYard. Daje użytkownikowi stosunkowo dużą kontrolę nad aplikacją w porównaniu do innych rozwiązań Platform as a Service. Umożliwia modyfikacje plików konfiguracyjnych serwera JBoss, dodawanie do niego nowych modułów. Pozwala także na tworzenie własnych kartridżów, które mogą uruchomić dowolną aplikację zgodną z systemem Linux. 


%---------------------------------------------------------------------------

\section{Proponowane ulepszenia}
\label{sec:ulepszenia}

Utworzony przez nas przykładowy system mający na celu prezentację mechanizmów bezpieczeństwa związanych z architekturą zorientowaną na usługi może stanowić punkt wyjścia do wprowadzenia omawianych rozwiązań w systemach zbudowanych w oparciu o SOA. Należy jednak pamiętać, że rozważany w niniejszej pracy model jest modelem uproszczonym i nie uwzględnia wszystkich możliwych sytuacji, które mogą wystąpić w rzeczywistych implementacjach SOA. Co więcej, nie wszystkie decyzje projektowe okazały się być trafnymi. W rozdziale tym zostaną przedstawione propozycje możliwych ulepszeń stworzonego systemu.

Pierwsza proponowana modyfikacja systemu dotyczy jego architektury. Mogłaby ona lepiej odpowiadać rzeczywistej architekturze systemów business to business, gdyby każda domena bezpieczeństwa posiadała własną usługę Security Token Service. W naszym systemie istnieje tylko jeden STS, a wystawionym przez niego asercjom ufa usługa uruchomiona w innej domenie bezpieczeństwa. Było to jedno z założeń projektowych, ustalonych na wczesnym etapie prac. Po dodaniu dodatkowego STS, należałoby wprowadzić federację w zarządzaniu tożsamościami(federated identity management) pomiędzy domenami bezpieczeństwa współpracujących ze sobą usług. Dzięki temu, token wystawiony przez STS w jednej domenie bezpieczeństwa mógłby w dalszym ciągu służyć do dostępu do usługi znajdującej się w domenie obsługiwanej przez inny STS.

Kolejny problem wynika z faktu, że SOA dopuszcza przetwarzanie wiadomości przez węzły pośrednie, a także umożliwia grupowanie wielu usług w usługę wyższego rzędu. W niektórych scenariuszach użycia kilka części wiadomości może mieć być przeznaczonych dla różnych odbiorców. Co za tym idzie, mechanizmy zabezpieczające wyłącznie kanał transmisji, mogą być niewystarczające do zapewnienia poufności i integralności danych przeznaczonych dla innych usług niż odbiorca wiadomości. W celu zabezpieczenia takiej komunikacji, konieczne jest zastosowanie mechanizmów zapewniających bezpieczeństwo na poziomie wiadomości. Dla formatu XML umożliwia to standard WS-Security w połączeniu ze XML Encryption i XML Signature. Podobne standardy dla wiadomości w formacie JSON są dopiero w fazie szkiców\cite{JWE14}\cite{JWS14} i nie są jeszcze wspierane przez biblioteki.

Ostatnie proponowane ulepszenie związane jest z jednym z głównych celów SOA. Jest nim zapewnienie zdolności do współpracy pomiędzy aplikacjami(interoperability). Mnogość zagadnień związanych z bezpieczeństwem w niektórych wypadkach powoduje, że integracja systemów staje się trudna. Wprowadzenie mechanizmów zarządzania tożsamościami i wydzielenie osobnego serwisu zarządzającego uwierzytelnianiem i autoryzacją częściowo rozwiązuje ten problem, przenosząc znaczną część odpowiedzialności dotyczącej bezpieczeństwa z dostawców usług do STS. Mimo to, poszczególne usługi wciąż mogą różnić się od siebie wymaganiami bezpieczeństwa. Przykładowo, usługa może wymagać określonego sposobu uwierzytelniania w STS lub wykorzystania przez XML Encryption konkretnego algorytmu szyfrowania. Jednym z możliwych rozwiązań tego problemu jest zastosowanie mechanizmów polityk bezpieczeństwa serwisów. Standard WS-Policy i jego rozszerzenie dotyczące bezpieczeństwa - WS-SecurityPolicy mają na celu rozwiązanie problemów związanych z możliwością komunikacji serwisów, umożliwiając negocjację parametrów bezpieczeństwa. Dodatkowo standardy te pozwalają na standaryzację wymogów bezpieczeństwa oraz zarządzanie nimi niezależnie od kodu usługi, np. poprzez dodanie ich do UDDI. Docelowo cały proces negocjacji, pozyskania tokenu bezpieczeństwa oraz szyfrowania i podpisywania określonych fragmentów wiadomości powinien być dostarczany przez biblioteki, podobnie jak w przypadku SSL/TLS.  Niestety wybrane przez nas technologie nie oferują takich możliwości.


%---------------------------------------------------------------------------
