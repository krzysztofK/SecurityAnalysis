\chapter{Analiza zastosowań tworzonego systemu}
\label{cha:zastosowania}

{\it

Niniejszy rozdział przedstawia możliwości zastosowania proponowanych rozwiązań w rzeczywistych systemach o architekturze SOA oraz wynikające z tego korzyści. Wykorzystanie mechanizmów systemów zarządzania tożsamościami pozwala zapewnić bezpieczeństwo aplikacji o architekturze zorientowanej na usługi. Zastosowany standard SAML i definiowane przy jego użyciu mechanizmy stanowią rozwiązanie dobrze dopasowane do specyfiki systemów o architekturze SOA. Elastyczność i skalowalność proponowanego podejścia sprawiają, że możliwe jest jego efektywne wdrożenie w rzeczywistym środowisku systemu bazującego na architekturze SOA.

Rozdział opisuje również potencjalne ulepszenia proponowanego rozwiązania. Zwiększenie poziomu bezpieczeństwa i rozszerzenie funkcjonalności systemu mogłoby być możliwe np. dzięki: wprowadzeniu wielu usług uwierzytelniających STS powiązanych relacją zaufania, zastosowaniu szyfrowania na poziomie fragmentów przesyłanych wiadomości, możliwości definiowania parametrów bezpieczeństwa przy użyciu polityk lub integracji usług STS(używanych przez serwisy webowe) z usługami IdP(używanymi przez aplikacje przeglądarkowe).

}

%---------------------------------------------------------------------------

\section{Możliwości wykorzystania środowiska w praktyce}
\label{sec:wykorzystanieWPraktyce}

	System powstały w wyniku wykonanych prac projektowych stanowi przykład wykorzystania podejścia opartego o zarządzanie tożsamościami jako metody zapewnienia bezpieczeństwa dostępu do aplikacji w architekturze SOA. Przygotowana implementacja dowodzi, że  istnieje możliwość oparcia funkcjonowania mechanizmów bezpieczeństwa aplikacji w architekturze SOA na rozwiązaniach dostarczanych przez systemy zarządzania tożsamościami. 

	Wynikiem prac implementacyjnych jest system wykorzystujący standard SAML jako realizację założeń systemów zarządzania tożsamościami. Przygotowany system dostarcza przykładów zastosowania specyfikacji SAML jako metody uwierzytelniania klientów usług sieciowych. Wykorzystanie  SAML w procesie uwierzytelniania umożliwia implementację mechanizmów bezpieczeństwa dla usług opartych o różne standardy dostarczania usług sieciowych(np. SOAP i REST). Do architektury przykładowego systemu wprowadzono również moduł magistrali usług - ESB, będący charakterystycznym elementem systemów opartych o architekturę SOA. Zaproponowana implementacja modułu magistrali usług dostarcza funkcjonalności przetwarzania wiadomości z dołączonymi tokenami bezpieczeństwa - asercjami języka SAML. Dzięki zastosowaniu mechanizmów bezpieczeństwa bazujących na specyfikacji SAML możliwe było osiągnięcie funkcjonalności jednokrotnego uwierzytelniania klienta dla różnych usług systemu o architekturze typu SOA. Mechanizm jednokrotnego uwierzytelniania może być wykorzystany podczas budowania procesu biznesowego odwołującego się do różnych usług sieciowych.

	Zastosowanie mechanizmów systemów zarządzania tożsamościami w kontekście aplikacji w architekturze SOA jest rozwiązaniem posiadającym wiele zalet. Może stanowić sposób standaryzacji implementacji mechanizmów zapewniania bezpieczeństwa aplikacji. Opiera się na oddelegowaniu odpowiedzialności związanych z bezpieczeństwem aplikacji do specjalizowanych usług. Dzięki zastosowaniu standardu SAML możliwe jest tworzenie relacji zaufania pomiędzy różnymi usługami uwierzytelniania i autoryzacji. Dzięki temu klient aplikacji może w transparentny dla siebie sposób korzystać z usług w obrębie różnych domen bezpieczeństwa. Zastosowanie zarządzania tożsamościami jest podejściem elastycznym, skalowalnym, umożliwiającym efektywne rozszerzanie na kolejne elementy infrastruktury systemu. Oddelegowanie odpowiedzialności związanych z uwierzytelnianiem i autoryzacją do odrębnej, specjalizowanej usługi zwiększa poziom bezpieczeństwa systemu. Wykorzystanie specyfikacji SAML umożliwia realizację procesu jednokrotnego uwierzytelniania klienta korzystającego z wielu usług sieciowych.

	Proces zapewnienia bezpieczeństwa dostępu do systemu opartego o architekturę SOA może być realizowany z wykorzystaniem mechanizmów zarządzania tożsamościami. Zaproponowane rozwiązanie wykorzystujące standard SAML jest podejściem dobrze dopasowanym do specyfiki i wymagań systemów o architekturze SOA.

%---------------------------------------------------------------------------

\section{Użyteczność zastosowanych technologii w systemach udostępniania usług}
\label{sec:uzytecznosc}

Użyteczność zastosowanych technologii w systemach udostępniania usług

%---------------------------------------------------------------------------

\section{Proponowane ulepszenia}
\label{sec:ulepszenia}

Proponowane ulepszenia

%---------------------------------------------------------------------------
