\chapter{Wstęp}
\label{cha:wstep}

%---------------------------------------------------------------------------

\section{Motywacja}
\label{sec:motywacja}

Zagadnienia związane z zapewnieniem bezpieczeństwa dostępu do aplikacji stanowią jeden z istonych problemów projektowania i implementacji systemów o charakterze rozpropszonym. Systemy tego typu dostarczają często skomplikowanych funkcjonalności i udostępniają poufne dane lub informacje o wysokiej wartości, takie jak np. wiadomości o stanie zdrowia, dane urzędowe lub informacje bankowe. Wymusza to stosowanie technik ograniczania dostępu do danych i funkcjonalności systemów informatycznych.
\\
\indent
Systemy rozproszone budowane są niejednokrotnie w oparciu o złożoną architekturę wielu komponentów i powiązań pomiędzy nimi. W środowiskach tego typu duże wyzwanie stanowi zarządzanie informacjami uwierzytelniającymi użytkownika systemu oraz przekazywanie tych informacji pomiędzy składowymi systemu. Często zachodzi potrzeba integracji pomiędzy różnymi aplikacjami. Dostarcza to kolejnych wyzwań w zakresie zapewnienia poprawnych interakcji pomiędzy systemami.
\\
\indent
Jednym z głównych czynników wpływających na skuteczność wdrażanych mechanizmów bezpieczeństwa aplikacji jest sposób ich wykorzystywania przez użytkowników. Zarządzanie przez użytkownika własnymi danymi uwierzytelniającymi jest ważnym elementem wpływającym na efektywność systemu zabezpieczeń aplikacji. Jednocześnie konieczność zapamiętywania rosnącej liczby haseł dla różnych usług stwarza zagrożenie związane ze stosowaniem słabych lub podobnych haseł dla różnych usług użytkownika. Jedną z koncepcji rozwiązania tego problemu jest mechanizm jednokrotnego uwierzytelniania, dzięki któremu użytkownik podając jedno bardziej skomplikowane hasło otwiera sobie drogę do korzystania z wielu usług. Implementacja tej koncepji stawia kolejne wyzwania związane z wymianą informacji uwierzytelniających pomiędzy różnymi systemami i tworzeniem relacji zaufania pomiędzy  aplikacjami.

%---------------------------------------------------------------------------

\section{Zarys dziedziny problemu}
\label{sec:zarysDziedzinyProblemu}

W rodziale~\ref{cha:analizaWymagan} przedstawiono podstawowe informacje dotyczące wymagań.... Alvis~\cite{Alvis2011} jest językiem 


















