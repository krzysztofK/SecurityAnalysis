\chapter{Wstęp}
\label{cha:wstep}

%---------------------------------------------------------------------------

\section{Motywacje}
\label{sec:motywacje}

Zagadnienia związane z zapewnieniem bezpieczeństwa dostępu do aplikacji stanowią jeden z istotnych problemów projektowania i implementacji systemów o charakterze rozproszonym. Systemy tego typu dostarczają często skomplikowanych funkcjonalności i udostępniają poufne dane lub informacje o wysokiej wartości, takie jak np. wiadomości o stanie zdrowia, dane urzędowe lu+-b informacje bankowe. Wymusza to stosowanie technik ograniczania dostępu do danych i funkcjonalności systemów informatycznych.

Istnieje wiele koncepcji opisujących podejście do tworzenia systemów rozproszonych. Jednym z najbardziej istotnych założeń projektowych i architektonicznych jest koncepcja architektury zorientowanej na usługi(ang. \textit{Service-oriented Architecture}). Systemy realizowane w paradygmacie SOA składają się ze zbioru różnorodnych usług, które można ze sobą łączyć w celu osiągnięcia określonego celu. Systemy rozproszone budowane zgodnie z koncepcją SOA charakteryzują się złożoną architekturą wielu komponentów i powiązań pomiędzy nimi. W środowiskach tego typu duże wyzwanie stanowi zarządzanie informacjami uwierzytelniającymi użytkownika systemu oraz przekazywanie tych informacji pomiędzy składowymi systemu. Często zachodzi potrzeba integracji pomiędzy różnymi aplikacjami. Dostarcza to kolejnych wyzwań w zakresie zapewnienia poprawnych interakcji pomiędzy usługami.

Jednym z głównych czynników wpływających na skuteczność wdrażanych mechanizmów bezpieczeństwa aplikacji jest sposób ich wykorzystywania przez użytkowników. Zarządzanie przez użytkownika własnymi danymi uwierzytelniającymi jest ważnym elementem wpływającym na efektywność systemu zabezpieczeń aplikacji. Jednocześnie konieczność zapamiętywania rosnącej liczby haseł dla różnych usług stwarza zagrożenie związane ze stosowaniem słabych lub podobnych haseł dla różnych usług użytkownika. Jedną z koncepcji rozwiązania tego problemu jest mechanizm jednokrotnego uwierzytelniania, dzięki któremu użytkownik podając jedno bardziej skomplikowane hasło otwiera sobie drogę do korzystania z wielu usług. Implementacja tej koncepcji stawia kolejne wyzwania związane z wymianą informacji uwierzytelniających pomiędzy różnymi systemami i tworzeniem relacji zaufania pomiędzy  aplikacjami.

%---------------------------------------------------------------------------

\section{Cele pracy}
\label{sec:celePracy}

	Niniejsza praca dotyka problemu jednokrotnego uwierzytelniania dla korzystania z wielu aplikacji. Dla tego typu zastosowań analizowane jest użycie różnych technik i metodologii pozwalających na gromadzenie i wymianę informacji na temat użytkownika pomiędzy aplikacjami. W wyniku wymiany tych informacji powinno być możliwe wdrożenie koncepcji jednokrotnego uwierzytelniania użytkownika.

	Najprostszym przypadkiem zastosowania koncepcji jednokrotnego uwierzytelniania jaki powinien zostać zrealizowany jest implementacja mechanizmu jednokrotnego logowania i wylogowywania użytkownika w różnych aplikacjach udostępnianych w przeglądarce internetowej. Najbardziej interesującym przypadkiem z punktu widzenia tematu pracy jest przeniesienie koncepcji jednokrotnego uwierzytelniania znanych ze standardowych aplikacji webowych na proces uwierzytelnienia klientów usług webowych.

	Głównym zadaniem pracy jest analiza mechanizmów zarządzania danymi uwierzytelniającymi w architekturze zorientowanej na usługi. Systemy realizowane w paradygmacie SOA składają się ze zbioru różnorodnych usług, które można ze sobą łączyć w celu osiągnięcia określonego celu. Każda z usług może posiadać własne mechanizmy uwierzytelniania i autoryzacji. Standaryzacja tych mechanizmów znacznie ułatwi proces integracji pomiędzy usługami. Szczególnie duże znaczenie ma to to w procesie komunikacji pomiędzy usługami różnych dostawców. 

	Istnieje kilka koncepcji związanych ze standaryzacją uwierzytelniania i autoryzacji użytkowników systemów informatycznych. Niniejsza praca kładzie główny nacisk na rozwiązania określane jako cyfrowe zarządzanie tożsamościami(Digital Identity Management). Koncepcja ta opiera się na oddelegowaniu odpowiedzialności związanych z uwierzytelnianiem użytkowników do usługi typu IdP(Identity Provider). Rozwiązanie to ułatwia ujednolicenie procesów uwierzytelniających w obrębie złożonego systemu.

	Koncepcje te powinny być przedstawione poprzez implementację przykładowego systemu(Prove of Concept) obsługi zamówień sklepu internetowego. Na system ten składa się kilka usług odpowiedzialnych za poszczególne etapy obsługi zlecenia. Poszczególny komponenty systemu powinny pracować w różnych domenach a integracja pomiędzy nimi powinna uwzględniać aspekty związane z bezpieczeństwem komunikacji pomiędzy usługami w oparciu o mechanizmy jednokrotnego uwierzytelniania.

%---------------------------------------------------------------------------

\section{Osiągnięcia pracy}
\label{sec:osiagnieciaPracy}

	Niniejsza praca przedstawia różne podejścia do zagadnienia zapewniania bezpieczeństwa do zasobów aplikacji. Przeprowadzone prace projektowe koncentrowały się na rozwiązaniach wprowadzonych przez koncepcję systemów zarządzania tożsamościami. Praca prezentuje architekturę i założenia dla systemów tego typu. Przedstawia również najbardziej popularne standardy realizujące założenia systemów zarządzania tożsamościami.

	Przeprowadzone prace implementacyjne wskazują zastosowania koncepcji systemów zarządzania tożsamościami dla rozwiązań gwarantowania bezpieczeństwa dostępu do aplikacji różnych typów, architektur i technologii. Punktem wyjścia dla wykonanych prac było wdrożenie mechanizmu jednokrotnego uwierzytelniania dla dostępu do aplikacji udostępnianych poprzez przeglądarkę internetową. Wykorzystanie rozwiązań wprowadzonych przez systemy zarządzania tożsamości dla tego typu zastosowań jest scenariuszem rozpoznanym i opisanym w licznych dostępnych materiałach.

	Najważniejszym celem osiągniętym w ramach wykonanych prac było wdrożenie koncepcji systemów zarządzania tożsamościami dla przykładowego systemu realizowanego zgodnie z założeniami koncepcji architektury zorientowanej na usługi. Mechanizmy wprowadzane przez systemy zarządzania tożsamościami zostały wykorzystane w celu zagwarantowania bezpieczeństwa dostępu do usług udostępnianych w ramach architektury SOA. W tym celu opracowano metody uwierzytelniania i autoryzacji klientów serwisów w oparciu o dane identyfikacyjne dostarczane przez systemy zarządzania tożsamościami. Opracowane zostały mechanizmy zabezpieczania dostępu do aplikacji dla różnych standardów dostarczania usług. Przedstawiono propozycje rozwiązań dla zarządzania cyfrowymi tożsamości w procesie przetwarzania żądań w architekturze zorientowanej na usługi uwzględniając jej charakterystyczne elementy. Wykorzystanie mechanizmów uwierzytelniania i autoryzacji opartych o rozwiązania dostarczane przez systemy zarządzania tożsamościami zostało również przedstawione w zastosowaniach związanych z modelowaniem procesów biznesowych.
