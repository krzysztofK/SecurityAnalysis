\chapter{Wstęp}
\label{cha:wstep}

%---------------------------------------------------------------------------

\section{Motywacja}
\label{sec:motywacja}

Zagadnienia związane z zapewnieniem bezpieczeństwa dostępu do aplikacji stanowią jeden z istonych problemów projektowania i implementacji systemów o charakterze rozpropszonym. Systemy tego typu dostarczają często skomplikowanych funkcjonalności i udostępniają poufne dane lub informacje o wysokiej wartości, takie jak np. wiadomości o stanie zdrowia, dane urzędowe lub informacje bankowe. Wymusza to stosowanie technik ograniczania dostępu do danych i funkcjonalności systemów informatycznych.

Systemy rozproszone budowane są niejednokrotnie w oparciu o złożoną architekturę wielu komponentów i powiązań pomiędzy nimi. W środowiskach tego typu duże wyzwanie stanowi zarządzanie informacjami uwierzytelniającymi użytkownika systemu oraz przekazywanie tych informacji pomiędzy składowymi systemu. Często zachodzi potrzeba integracji pomiędzy różnymi aplikacjami. Dostarcza to kolejnych wyzwań w zakresie zapewnienia poprawnych interakcji pomiędzy systemami.

Jednym z głównych czynników wpływających na skuteczność wdrażanych mechanizmów bezpieczeństwa aplikacji jest sposób ich wykorzystywania przez użytkowników. Zarządzanie przez użytkownika własnymi danymi uwierzytelniającymi jest ważnym elementem wpływającym na efektywność systemu zabezpieczeń aplikacji. Jednocześnie konieczność zapamiętywania rosnącej liczby haseł dla różnych usług stwarza zagrożenie związane ze stosowaniem słabych lub podobnych haseł dla różnych usług użytkownika. Jedną z koncepcji rozwiązania tego problemu jest mechanizm jednokrotnego uwierzytelniania, dzięki któremu użytkownik podając jedno bardziej skomplikowane hasło otwiera sobie drogę do korzystania z wielu usług. Implementacja tej koncepcji stawia kolejne wyzwania związane z wymianą informacji uwierzytelniających pomiędzy różnymi systemami i tworzeniem relacji zaufania pomiędzy  aplikacjami.

%---------------------------------------------------------------------------

\section{Zarys dziedziny problemu}
\label{sec:zarysDziedzinyProblemu}

Niniejsza praca dotyka problemu jednokrotnego uwierzytelniania dla korzystania z wielu aplikacji. Dla tego typu zastosowań analizowane jest użycie różnych technik i metodologii pozwalających na gromadzenie i wymianę informacji na temat użytkownika pomiędzy aplikacjami. W wyniku wymiany tych informacji powinno być możliwe wdrożenie koncepcji jednokrotnego uwierzytelniania użytkownika.

Najprostszym przypadkiem zastosowania koncepcji jednokrotnego uwierzytelniania jest implementacja mechanizmu jednokrotnego logowania i wylogowywania użytkownika w różnych aplikacjach udostępnianych w przeglądarce internetowej. Najbardziej interesującym przypadkiem z punktu widzenia tematu pracy jest przeniesienie koncepcji jednokrotnego uwierzytelniania znanych ze standardowych aplikacji webowych na proces uwierzytelnienia klientów usług webowych.

Głównym zadaniem pracy jest analiza mechanizmów zarządzania danymi utwierzytelniającymi w architekturze zorientowanej na usługi(SOA - Service-oriented Architecture). Systemy realizowane w paradygmacie SOA składają się ze zbioru różnorodnych usług, które można ze sobą łączyć w celu osiagnięcia określonego celu. Każda z usług może posiadać własne mechanizmy uwierzytelniania i autoryzacji. Ustandaryzowanie tych mechanizmów znacznie ułatwi proces integracji pomiędzy usługami. Szczególnie duże znaczenie ma to to w procesie komunikacji pomiędzy usługami różnych dostawców. 

Istnieje kilka koncepcji związanych ze standaryzacją uwierzytelniania użytkowników systemów informatycznych. Niniejsza praca kładzie główny nacisk na rozwiązania określane jako cyfrowe zarządzanie tożsamościami(Digital Identity Management). Koncepcja ta opiera się na oddelegowaniu odpowiedzialności związanych z uwierzytelnianiem użytkwoników do usługi typu IdP(Identity Provider). Rozwiązanie to ułatwia ujednolicenie procesów uwierzytelniających w obrębie złożonego systemu.

Koncepcje te zrealizowano poprzez implementację przykładowego systemu(Prove of Concept) obsługi zamówień sklepu internetowego. Na system ten składa się kilka usług odpowiedzialnych za poszczególne etapy obsługi zlecenia. Poszczególny komponenty systemu powinny pracować w różnych domenach a integracja pomiędzy nimi powinna uwzględniać aspekty związane z bezpieczeństwem komunikacji pomiędzy usługami w oparciu o mechanizmy jednokrotnego uwierzytelniania.
