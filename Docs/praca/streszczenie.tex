\begin{abstract}

	Niniejsza praca przedstawia analizę narzędzi zarządzania mechanizmami bezpieczeństwa w aplikacjach opartych o architekturę zorientowaną na usługi(ang. \textit{Service-oriented Architecture}). W pracy opisano koncepcję architektury systemów opartych o udostępnianie usług. Praca prezentuje różne standardy zapewniania bezpieczeństwa dostępu do aplikacji, szczególnie te związane z uwierzytelnianiem i autoryzacją klientów usług sieciowych. Głównym przedmiotem zainteresowania niniejszej pracy jest zbadanie użyteczności różnych mechanizmów zapewniania bezpieczeństwa aplikacji i zaproponowanie rozwiązania najlepiej dopasowanego do specyfiki systemów o architekturze SOA.

	Przeprowadzona analiza narzędzi zapewniania bezpieczeństwa dostępu do aplikacji koncentruje się na rozwiązaniach opartych o koncepcję systemów zarządzania tożsamościami. Jest to podejście do zagadnienia bezpieczeństwa dostępu do usług oparte o pojęcie tożsamości użytkownika oraz definicje procesów zarządzania atrybutami związanymi z tożsamością. W pracy opisano podstawy kilku istniejących standardów realizujących idee systemów zarządzania tożsamościami. Implementację przykładowych mechanizmów zabezpieczania aplikacji oparto o jeden z tych standardów - specyfikację SAML(Security Assertion Markup Language). 

	Standard SAML pozwala na wymianę informacji o tożsamości użytkowników oraz umożliwia implementację procesu uwierzytelniania w oparciu o te informacje(w tym również procedurę jednokrotnego uwierzytelniania). W ramach prac projektowych zaimplementowane zostały przykładowe aplikacje ilustrujące zastosowania systemów zarządzania tożsamościami i specyfikacji SAML dla rozwiązań opartych o architekturę zorientowaną na usługi.

	Pierwszy rozdział niniejszej pracy opisuje jej motywacje; przedstawia cele stawiane przed wykonywanym projektem; prezentuje również osiągnięcia będące wynikiem wykonanych prac. W kolejnym rozdziale zawarte jest wprowadzenie do problematyki poruszanej w pracy. Rozdział zawiera opis koncepcji systemów zarządzania tożsamościami, wprowadza do podejścia jednokrotnego uwierzytelniania klientów aplikacji, prezentuje model architektury zorientowanej na usługi. Kolejny rozdział stanowi prezentację dostępnych rozwiązań użytecznych w kontekście uwierzytelniania klientów usług sieciowych. Zawiera opis standardów wykorzystywanych przez mechanizmy zapewniania bezpieczeństwa dostępu do usług. Przedstawia standardy realizujące założenia systemów zarządzania tożsamościami.

	Kolejne rozdziały przedstawiają efekty prac projektowych oraz przeprowadzonych analiz. Ta część pracy rozpoczyna się rozdziałem prezentującym propozycję architektury systemów opartych o model SOA wykorzystujących koncepcję systemów zarządzania tożsamościami w procesie uwierzytelniania użytkowników. Kolejny rozdział przedstawia technologie umożliwiające realizacje założeń architektonicznych opisanych we wcześniejszym fragmencie pracy. Następnie zawarto w pracy opis przykładowej implementacji wykorzystującej prezentowane technologie i proponowane założenia architektoniczne. Praca zakończona jest rozdziałami zawierającymi analizę zastosowań przygotowanych rozwiązań i wnioski wyciągnięte na podstawie wykonanych prac projektowych oraz wiedzy zdobytej na etapie analizy problematyki tematu pracy.

\end{abstract}
