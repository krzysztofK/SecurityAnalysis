\chapter{Analiza dziedziny problemu}
\label{cha:analizaDziedzinyProblemu}

{\it
	Niniejszy rozdział stanowi wprowadzenie do problematyki analizowanej i omawianej w przedstawionej pracy. U podstaw pracy leżą zagadnienia związane z  systemami zarządzania tożsamościami. Jest to koncepcja opisująca mechanizmy systemów magazynowania, przetwarzania i wykorzystywania tożsamości użytkowników na różnych etapach obsługi klientów. Celem stawianym przed systemami tego typu jest zapewnienie bezpieczeństwa wymiany informacji a także zwiększenie zakresu funkcjonalności oferowanych użytkownikom. Rozdział zawiera definicję i opis funkcjonowania systemów zarządzania tożsamościami. Wprowadza również pojęcie federacji tożsamości - połączenia pomiędzy różnymi tożsamościami tego samego użytkownika. 

	Rozdział porusza tematykę zastosowania mechanizmów jednokrotnego uwierzytelniania dla dostępu do różnych aplikacji. Opisuje architekturę najczęstszych rozwiązań wykorzystujących to podejście do przydzielania praw dostępu do zasobów systemów informatycznych. 

	Niniejsza praca koncentruje się na analizie zastosowań koncepcji systemów zarządzania tożsamości w architekturze zorientowanej na usługi. Kluczowym elementem rozwiązań opartych o architekturę tego typu są serwisy - funkcjonalności udostępnianie  klientom, zamknięte w ramach dobrze zdefiniowanych interfejsów. Rozdział prezentuje elementy architektury zorientowanej na usługi, przedstawia strukturę powiązań i zależności pomiędzy nimi, opisuje sposób funkcjonowania systemów realizowanych w oparciu o architekturę tego typu. 

}

%---------------------------------------------------------------------------

\autsection{Podstawowe aspekty związane z bezpieczeństwem systemów informatycznych}{Krzysztof Wilaszek}
\label{sec:aspektyBezpieczenstwa}

Zagadnienie zapewnienia bezpieczeństwa w kontekście systemów komputerowych odnosi się do metod ochrony informacji zawartych w systemie informatycznym przed nieupoważnionym dostępem oraz metod gwarantowania poufności i integralności przetwarzanych danych. Dotyczy również problematyki zapewniania dostępności i wiarygodności funkcjonalności dostarczanych przez systemy informatyczne. Zadaniem stawianym przed mechanizmami zapewniania bezpieczeństwa systemów informatycznych jest minimalizacja prawdopodobieństwa naruszenia któregoś z czynników warunkujących poziom bezpieczeństwa systemu. 

Istnieją 4 podstawowe metody zabezpieczania systemów informatycznych \cite{Russell91}:

\begin{itemize}
	\item Kontrola dostępu do systemu, której celem jest umożliwienie korzystania z zasobów systemu jedynie uwierzytelnionym i uprawnionym użytkownikom. W obrębie tej metody mieszczą się również funkcjonalności systemu wymuszające na użytkownikach zgodność z przyjętymi procedurami zwiększania bezpieczeństwa systemu. Obejmuje ona również działania związane z monitorowaniem zdarzeń w systemie - szczególnie tych związanych z pozyskiwaniem dostępu do zasobów systemu. 
	\item Kontrola dostępu do danych - polegająca na definiowaniu uprawnień użytkowników do korzystania z informacji zawartych w systemie pod określonymi warunkami.
	\item Administracja systemu i jego aspektów bezpieczeństwa - obejmuje działania administracyjne takie jak nadzór nad realizacją założeń przyjętej polityki bezpieczeństwa lub analizę zagrożeń bezpieczeństwa systemu i redukcję ryzyka wystąpienia tych zagrożeń. 
	\item Uwzględnienie zagrożeń bezpieczeństwa systemu na etapie projektowania architektury aplikacji. 
\end{itemize} 
		
\subsection{Kontrola dostępu do systemów informatycznych}

	Podstawą kontroli dostępu do systemów informatycznych jest proces identyfikacji i uwierzytelniania użytkowników. W procesie tym klient aplikacji przedstawia swoją tożsamość oraz dowodzi jej. Istnieją różne typy potwierdzania tożsamości użytkowników aplikacji. Najbardziej popularnym sposobem jest weryfikacja wiedzy użytkownika dotyczącej jego atrybutów bezpieczeństwa(najbardziej powszechnie wykorzystywanym podejściem jest zastosowanie haseł). Inne podejście zakłada konieczność posiadania atrybutu potwierdzającego tożsamość użytkownika. Atrybutem tym może być klucz, identyfikator, karta(np. bankomatowa) czy token. Istnieją również metody oparte o cechy użytkownika np. odcisk palca, obraz siatkówki oka, głos, podpis. Cechy zgromadzone w systemie biometrycznym porównywane są z cechami uwierzytelnianego użytkownika i na tej podstawie przydzielane są prawa dostępu do zasobów systemu.

	Potwierdzona tożsamość użytkownika aplikacji wykorzystywana jest w procesie określania jego uprawnień do poszczególnych zasobów systemu. Często systemy wykorzystują informacje o tożsamości użytkownika podczas logowania zdarzeń zachodzących w systemie. Najczęściej systemy przypisują użytkowników do grup określających uprawnienia swoich członków.

%---------------------------------------------------------------------------


\autsection{Koncepcja cyfrowego zarządzania tożsamościami(IdM)}{Krzysztof Wilaszek}
\label{sec:konceptcjaIdM}

	Zarządzanie tożsamościami(Identity Management) jest podejściem do zagadnienia zapewnienia bezpieczeństwa dostępu do aplikacji w oparciu o dane identyfikujące użytkownika(Identity). Pojęcie danych identyfikujących może być zdefiniowane jako ,,informacje o jednostce pozwalające na identyfikację tej jednostki dla pewnej dziedziny zastosowań''\cite{Itu09}. W kontekście systemów zarządzania cyfrowymi tożsamościami dane identyfikujące mogą dotyczyć nie tylko osób ale również na przykład komponentów programowych\cite{Bertino11}. Zgodnie z rekomendacją Y.2720 w skład danych identyfikujących wchodzą: identyfikator jednostki, dane uwierzytelniające jednostkę oraz atrybuty opisujące jednostkę\cite{Itu09}.

	Elisa Bertino i Kenji Takahashi w książce ,,Identity Management'' definiują cele zarządzania tożsamościami jako ,,utrzymanie integralności danych identyfikujących w trakcie ich użytkowania w celu udostępniania tych danych i powiązanych z nimi informacji w sposób bezpieczny i chroniący prywatność użytkowników''\cite{Bertino11}.
	 
	\subsection{Role w koncepcji systemów zarządzania tożsamościami}

		Systemy zarządzania tożsamościami charakteryzują się rozdzieleniem odpowiedzialności związanych z dostarczaniem funkcjonalności oraz zadań związanych z zapewnieniem bezpieczeństwa dostępu do aplikacji. Usługi uwierzytelniania i autoryzacji oraz funkcjonalności systemów dostarczane są dla jednostek - składowych systemu zarządzania tożsamościami - zazwyczaj użytkowników.
		Dane identyfikacyjne jednostek są gromadzone i wykorzystywane w trakcie korzystania z aplikacji. Dane osobowe mogą obejmować informacje związane z dokumentami tożsamości(numery dowodu osobistego, paszportu), dane bankowe, biometryczne oraz informacje związane z przebiegiem interakcji jednostki z systemem. Dane identyfikacyjne jednostek powinny być przetwarzane i gromadzone w sposób zapewniający  ochronę przed niewłaściwym użyciem. Nadużycia danych osobowych jednostek mogą prowadzić do istotnych strat użytkowników aplikacji.
		Elementem przeprowadzającym weryfikację tożsamości jednostek jest usługa ,,Identity Provider''(IdP). Usługa ta odpowiedzialna jest za przyporządkowywanie jednostkom atrybutów opisujących tożsamość, tworzenie powiązań pomiędzy różnymi atrybutami jednostki oraz tworzenie asercji zawierających informacje o atrybutach jednostek. 

		Usługa 'Identity Provider' może współpracować z innymi usługami tego typu dołączając do danych uwierzytelniających informacje udostępniane przez inne zaufane usługi IdP. Wykorzystanie danych uwierzytelniających dostarczanych przez inne usługi uwierzytelniające wymaga wprowadzenia procesu zapewnienia wiarygodności otrzymywanych danych. Proces ten może opierać się na przypisywaniu miary wiarygodności do atrybutów uwierzytelniających. Wyznaczenie wartości tej miary może być na przykład oparte o ocenę stopnia zaawansowania mechanizmów weryfikacji danych uwierzytelniających wykorzystywanych przez usługę dostarczającą dany atrybut tożsamości.

		Dostawcy usług wykorzystujący infrastrukturę systemów zarządzania tożsamościami przed zezwoleniem na dostęp do swoich zasobów zlecają usługom typu IdP przeprowadzenie procesu uwierzytelniania klienta na podstawie otrzymanych danych uwierzytelniających. Dostawcy usług powinni mieć możliwość deklaracji wymaganego poziomu skuteczności zabezpieczeń wykorzystywanych w procesie autoryzacji dostępu do określonych zasobów. Poziom ten może być różny w zależności od rodzaju zasobu.

		Przedstawiona w książce ,,Identity Management Concepts, Technologies, and Systems'' terminologia wprowadza również pojęcie jednostki nadzorującej\cite{Bertino11}. Jest to najczęściej instytucja upoważniona prawnie do nadzoru nad procesami przetwarzania i przechowania danych osobowych lub wglądu  w informacje o charakterze poufnym.

	\subsection{Relacje między rolami w systemach zarządzania tożsamościami}

		\begin{figure}[h]
			\centering
			\includegraphics[width=15cm]{img/idmRelations.jpg}
			\caption{Relacje miedzy rolami w systemach zarządzania tożsamościami}
			\label{Relacje miedzy rolami w systemach zarządzania tożsamościami}
		\end{figure}

		Często stosowanym modelem jest infrastruktura złożona z jednej usługi typu ,,Identity Provider'' oraz wielu usług funkcjonalnych opierających na niej swoje mechanizmy zabezpieczeń. Dzięki takiemu rozwiązaniu dostawcy usług mogą skoncentrować się na tworzeniu funkcjonalności stanowiących istotę aplikacji - odpowiedzialności związane z zapewnieniem bezpieczeństwa dostępu delegowane są do usługi IdP. Specjalizowana usługa uwierzytelniania może dostarczać bardziej zaawansowanych zabezpieczeń. Dzięki realizacji tego modelu użytkownicy nie muszą zarządzać wieloma danymi uwierzytelniającymi dla różnych usług - dostęp do wielu serwisów gwarantowany jest przy użyciu tych samych danych. Wprowadza to jednak zagrożenia związane z centralizacją dostępu do różnych usług.

	\subsection{Federated Identity Management}

		Najczęściej użytkownicy korzystają nie z jednej usługi lecz z szerokiej gamy różnych usług. W ramach każdej z usług istnieją odrębne dane uwierzytelniające. Podejście ,,Federated Identity Management'' umożliwia tworzenie powiązań pomiędzy tożsamościami użytkownika w ramach różnych usług dzięki czemu dane uwierzytelniające każdej z sfederowanych usług mogą być wykorzystane w procesie uwierzytelniania dowolnej z usług.

	\subsection{Cykl życia tożsamości}

		Jednym z głównych zadań systemów zarządzania tożsamościami jest kontrola nad cyklem życia tożsamości. Autorzy książki ,,Identity Management: Concepts, Technologies and Systems'' opisują 4 etapy cyklu życiu tożsamości: tworzenie, użytkowanie, aktualizacja oraz wycofanie z użycia\cite{Bertino11}.

		Proces tworzenia cyfrowej tożsamości składa się z kilku kroków. Pierwszym z nich może być weryfikacja przedstawionych atrybutów tożsamości, wymagająca udowodnia przez jednostkę prawdziwości wprowadzanych danych. Następnie tworzone są dane uwierzytelniające. Ostatnim krokiem jest utworzenie tożsamości na podstawie otrzymanych danych oraz nadanie jednostce identyfikatora.

		Utworzona tożsamość może być wykorzystywana w różnych celach, np. zapewnienia wiarygodnej komunikacji lub w procesie jednokrotnego uwierzytelniania(ang. Single Sign-On).

		Systemy zarządzania tożsamościami powinny obsługiwać zmiany atrybutów tożsamości. Powinny aktualizować informacje o danych jednostek po zmianach wysyłając powiadomienia do usług przechowujących te dane, np. ,,Identity Provider''. Identyfikatory jednostek nie powinny podlegać zmianom. Systemy IdM muszą również usuwać tożsamości jeśli nie są już aktualne.

\autsection{Jednokrotne uwierzytelnianie}{Krzysztof Wilaszek}

	Książka ,,Identity Management: Concepts, Technologies and Systems'' definiuje jednokrotne uwierzytelnianie(ang. Single Sign-On) jako proces uwierzytelniania, w którym jednostka może wykorzystać wynik pojedynczego uwierzytelniania dla uzyskania dostępu do wielu niezależnych usług z ochroną dostępu\cite{Bertino11}. 

	Podstawą funkcjonowania mechanizmów jednokrotnego uwierzytelniania jest nawiązanie relacji zaufania pomiędzy dostawcami usług oraz serwisami typu ,,Identity Provider''. Po uwierzytelnieniu użytkownika w ramach jednej z usług objętych mechanizmem SSO dostęp do innej nie wymaga uwierzytelniania - dane uwierzytelniające są mapowane na dane niezbędne do uwierzytelnienia względem innej usługi oraz generowane są informacje pozwalające na uzyskanie dostępu do serwisu. Usługi korzystające z mechanizmu jednokrotnego uwierzytelniania powinny otrzymywać informacje kontekstowe o przebiegu procesu uwierzytelniania takie jak: wykorzystywane metody uwierzytelniania oraz sposób ochrony danych uwierzytelniających. Informacje te pozwalają na ocenę stopnia wiarygodności przeprowadzonego procesu uwierzytelniania.

	\subsubsection{Architektura systemów jednokrotnego uwierzytelniania}

		Implementacja mechanizmu jednokrotnego uwierzytelniania może opierać się o różne architektury. Autorzy książki ,,Identity Management: Concepts, Technologies and Systems''\cite{Bertino11} wymieniają następujące typy architektur systemów jednokrotnego uwierzytelniania:

		\begin{itemize}
		  \item Architektura oparta o brokery - architektura składająca się z punktu centralnego(serwera) oraz jednostek przez niego uwierzytelnianych. Serwer przydziela użytkownikom tokeny uwierzytelniające, dzięki którym możliwy jest dostęp do aplikacji. Przykładem architektury tego typu jest protokół Kerberos. 
		  \item Architektura oparta o agenty - architektura, w której w dostępie do każdej aplikacji pośredniczy agent uwierzytelniania. Jego rolą jest translacja pomiędzy metodą uwierzytelniania zastosowaną przez klienta a mechanizmami obsługiwanymi przez aplikację.
		  \item ,,Reverse proxy-based architecture'' - architektura wprowadzająca usługę proxy pośredniczącą w dostępie do wszystkich aplikacji objętych procedurą SSO. Moduł proxy dokonuje filtrowania przychodzących komunikatów - nieuwierzytelnione żądania zostają przekierowane, np. do serwera uwierzytelniającego.
		\end{itemize}
		  
		Procedura jednokrotnego uwierzytelniania pozwala na wygodniejszy sposób korzystania z aplikacji - znosi konieczność wielokrotnego wprowadzania danych identyfikujących. Użytkownik nie musi też zapamiętywać wielu haseł dla różnych usług. Wiele prostych haseł może zostać zastąpionych jedną bardziej wiarygodną metodą uwierzytelniania(np. z wykorzystaniem bardziej skomplikowanego hasła). Wadą wprowadzenia procedury jednokrotnego uwierzytelniania jest jednak centralizacja punktu dostępu do różnych aplikacji - złamanie zabezpieczeń otworzy drogę do wielu usług użytkownika.

%---------------------------------------------------------------------------

\autsection{Service-Oriented Architecture}{Krzysztof Wilaszek, Tomasz Wójcik}
\label{sec:soa}

	Architektura zorientowana na usługi(ang. Service-Oriented Architecture) to architektura systemów informatycznych oparta o strukturę luźno powiązanych, rozproszonych usług, które mogą być wielokrotnie wykorzystywane i łączone w celu realizacji wymagań stawianych aplikacji. Architektura zorientowana na usługi umożliwia integrację usług w złożone procesy\cite{Lawler08}. Usługi składające się na architekturę SOA udostępniają swoje  funkcjonalności w ramach zdefiniowanych interfejsów i są od siebie niezależne. Komunikacja pomiędzy serwisami odbywa się poprzez wywołania dostarczanych przez nie operacji\cite{Papazoglou07}. 

	Architektura SOA zaprojektowana została jako rozwiązanie wielu problemów pojawiających się w trakcie tworzenia systemów rozproszonych. Do problemów tych należą: integracja aplikacji, zarządzanie transakcjami, zapewnienia bezpieczeństwa wykonywanych operacji, różnorodność środowisk uruchamiania aplikacji\cite{Papazoglou07}.

	Jednym z głównych założeń architektury zorientowanej na usługi jest niezależność od technologii implementacji. Jest to możliwe dzięki wykorzystaniu standardowych interfejsów usług oraz metod komunikacji pomiędzy nimi. Usługi w architekturze SOA są autonomiczne, same przechowują swój stan. W architekturze SOA wszelkie funkcjonalności są udostępniane w postaci usług.
	
	\subsection{Model architektury zorientowanej na usługi}
	
		Model SOA definiuje charakterystyczne elementy architektury - komponenty, usługi i przetwarzanie informacji pomiędzy nimi - realizujące określone procesy biznesowe. Model ma budowę warstwową. Definicja warstw modelu ma na celu bardziej precyzyjne odwzorowanie pomiędzy wymaganiami biznesowymi, modelowaniem funkcjonalności a realizacją konkretnych rozwiązań dla stawianych wymagań. Kluczową cechą charakterystyczną dla architektury SOA jest wyraźny rozdział pomiędzy interfejsem opisującym usługę a jej implementacją. Model architektury SOA umożliwia wydzielenie części odpowiedzialnych za poszczególne typy zadań oraz upraszcza proces projektowania i tworzenia aplikacji opartych o architekturę zorientowaną na usługi\cite{Arsanjani07}. 

		\begin{figure}[h]
			\centering
			\includegraphics[width=15cm]{img/s3.jpg}
			\caption{Warstwy modelu architektury zorientowanej na usługi, źródło - \textit{http://www.ibm.com/developerworks/library/ar-archtemp/}}
			\label{soaModel}
		\end{figure}
		
		Model architektury SOA składa się z dwóch typów warstw. Warstwy oznaczone na rysunku przy pomocy bloków poziomych ilustrują cechy funkcjonalne rozwiązań architektury SOA. Bloki pionowe przedstawiają cechy niefunkcjonalne odnoszące się do warstw funkcjonalnych. Model wprowadza rozgraniczenie pomiędzy dostawcami oraz konsumentami usług. Model zakłada powiązanie relacjami biznesowymi pomiędzy obydwoma grupami. Wyraźne rozdzielenie pomiędzy dostawcami i konsumentami usług pozwala określić kompetencje i wymagania stawiane obydwu rolom w architekturze SOA. Do warstw przypisywanych dostawcom usług należą niższe warstwy modelu - serwisów, komponentów oraz warstwa operacyjna. Wyższe warstwy - serwisów, procesów biznesowych oraz konsumentów - przypisane są konsumentom usług. 

		Cele stawiane tworzonej architekturze SOA wyznaczane są przez zestaw wymagań funkcjonalnych i niefunkcjonalnych stawianych systemom. Wśród wymagań niefunkcjonalnych, jakie uwzględniane są w opisie architektury SOA można wyróżnić - bezpieczeństwo, dostępność, wiarygodność, łatwość zarządzania, skalowalność oraz wymagania dotyczące wydajności dostarczanych rozwiązań. Wymagania realizowane są poprzez udostępniane serwisy. Realizacja wymagań może być osiągnięta poprzez funkcjonowanie jednej warstwy modelu architektury SOA lub poprzez połączenie działania wielu warstw. Kluczowym aspektem implementacji rozwiązań opartych o architekturę SOA jest identyfikacja wymagań stawianych systemowi i odzwierciedlenie tych wymagań na poszczególne warstwy modelu aplikacji. 

		\subsubsection{Warstwy funkcjonalne modelu architektury SOA}


			Warstwa operacji obejmuje wszelkie aplikacje uruchamiane w środowisku architektury SOA, wspierające funkcje biznesowe systemu. W tej warstwie włączane są istniejące aplikacje, których zastosowanie pomaga w realizacji celów stawianych systemowi. Pozwala to na wielokrotne wykorzystanie opracowywanych rozwiązań\cite{Arsanjani07}. 

			Warstwa komponentów serwisowych składa się z elementów dostarczających realizacji usług. Komponenty serwisów zawierają definicję cech funkcjonalnych i niefunkcjonalnych udostępnianych usług. Komponenty dostosowują kontrakt usług opisany w warstwie serwisów do wykorzystanych w systemie mechanizmów realizacji i implementacji funkcjonalności. Zadaniem komponentów jest spełnienie wymagań dotyczących jakości usług. Zwiększają elastyczność procesu realizacji wymagań biznesowych poprzez umożliwienie łatwego komponowania usług realizujących wymagane funkcjonalności. Upraszczają techniczne aspekty dostarczania serwisów poprzez rozdzielenie interfejsu usług od mechanizmów implementacyjnych.

			Warstwa usług składa się ze wszystkich serwisów zdefiniowanych w ramach architektury SOA. Usługa jest zbiorem funkcjonalności biznesowych udostępnianych przez aplikację w ramach dobrze zdefiniowanych interfejsów. Warstwa serwisów udostępnia interfejsy usług w postaci opisu. Opis usługi obejmuje informacje potrzebne do jej wywołania. Warstwa zawiera kontrakty dostarczania i wykorzystywania serwisów pomiędzy dostawcami i odbiorcami usług. Dostęp do serwisów jest niezależny od zastosowanych mechanizmów implementacji i transportu danych. 

			Warstwa procesów biznesowych definiuje mechanizmy kompozycji usług udostępnianych w warstwie serwisów. Kompozycja serwisów jest połączeniem usług w celu osiągnięcia przy ich użyciu określonych funkcjonalności - realizacji określonych procesów biznesowych. Warstwa obejmuje reprezentację procesów oraz mechanizmy kompozycji usług. Określa standardy wymiany informacji pomiędzy uczestnikami procesu biznesowego. Procesy determinowane są przez wymagania stawiane funkcjonalnościom dostarczanym przez system. Warstwa pełni rolę głównego łącznika pomiędzy wymaganiami z perspektywy klientów biznesowych a realizacją oczekiwanych funkcjonalności po stronie dostawcy usług.

			Warstwa konsumenta usług obejmuje metody dostarczania serwisów do użytkowników zgodnie z ich preferencjami. Uniezależnia korzystanie z procesów biznesowych od zastosowania konkretnego kanału komunikacyjnego. Warstwa wprowadza wzorce zastosowania pojedynczego punktu dostępu do usług systemu oraz pojedynczego punktu prezentacji wiedzy o systemie.

		\subsubsection{Warstwy niefunkcjonalne modelu architektury SOA}

			Funkcjonowanie rozwiązań w oparciu o architekturę SOA jest w dużej mierze możliwe dzięki mechanizmom zdefiniowanym w warstwie integracji. Warstwa ta umożliwia kierowanie i transport żądań klientów usług do odpowiednich dostawców serwisów. Pozwala na integrację pomiędzy różnymi usługami poprzez zastosowanie mechanizmów komunikacji punkt-punkt oraz bardziej zaawansowanych technik sterowania ruchem komunikatów(ang. \textit{routowania}) i transformowania zawartości wiadomości. Warstwa odpowiada głównie za integrację działania warstw komponentów, serwisów oraz procesów biznesowych. Definiuje standardy rozdziału pomiędzy dostawcami i konsumentami usług(np. poprzez wprowadzenie warstwy pośredniczącej - magistrali usług ESB)\cite{Arsanjani07}.

			Warstwa gwarantowania jakości dostarczania usług odpowiada za realizację wymagań niefunkcjonalnych stawianych systemowi. W zakres odpowiedzialności przypisanych tej warstwie wchodzą monitorowanie i logowanie parametrów jakości udostępniania usług oraz sygnalizowanie przypadków niezgodności aktualnych parametrów opisu jakości z wymaganiami stawianymi systemowi opartemu o architekturę SOA. Dostarcza metody zapewniania jakości usług w kontekście wiarygodności, dostępności, łatwości zarządzania, skalowalności i bezpieczeństwa. 

			Warstwa architektury danych i inteligencji biznesowej zapewnia dostarczenie kluczowych mechanizmów strukturalizacji danych stosowanych w celu realizacji zadań systemów inteligencji biznesowej w oparciu o narzędzia typu magazyny danych(ang. \textit{Data warehouses}). Warstwa odpowiada za przechowywanie metadanych oraz informacji o strukturze dostępnych danych. 

			Kierowanie funkcjonowaniem systemu opartego o architekturę SOA możliwe jest dzięki metodom zdefiniowanym w ramach warstwy zarządzania. Warstwa może kontrolować działanie każdej innej warstwy modelu. Jej funkcjonowanie wpływa na zapewnienie parametrów jakości dostarczania usług. Odpowiada za kierowanie wydajnością działania systemu oraz zarządza mechanizmami bezpieczeństwa. 

	\subsection{Enterprise Service Bus} 

		W istniejących środowiskach dostarczania usług występuje często duża niejednorodność technologii, protokołów komunikacji i modeli wykorzystywanych przez różne serwisy. Jednym z rozwiązań tego problemu jest wprowadzenie warstwy pośredniczącej, która implementuje logikę pozwalającą na integrację różnorodnych usług. Rozwiązanie to stanowi podstawę dla koncepcji magistrali usług(ang. Enterprise Service Bus). 

		\begin{figure}[h]
			\centering
			\includegraphics[width=15cm,height=8cm]{img/esb.jpg}
			\caption{Schemat funkcjonowania magistrali usług}
			\label{Schemat funkcjonowania magistrali uslug}
		\end{figure}

		Magistrala ESB dostarcza funkcjonalności rozproszonego przetwarzania oraz integracji usług opartych o standardowe mechanizmy. Funkcje transportu i transformacji wiadomości pozwalają na komunikację pomiędzy niejednorodnymi i rozproszonymi usługami. Magistrala usług może zapewniać mechanizmy bezpieczeństwa dostępu do aplikacji, wiarygodności dostarczania danych oraz audytu komunikacji. ESB powinno pozwalać na przekazywanie informacji kontekstowych np. dotyczących transakcji lub bezpieczeństwa dostępu do usług.

		Mechanizmy ESB uniezależniają klientów usług od fizycznych właściwości dostarczania usługi. Dzięki zastosowaniu magistrali usług możliwe jest dokonywanie zmian po stronie usługi lub zastąpienie usługi inną bez wpływu na aplikację kliencką. 

	\subsection{Zarządzanie procesami biznesowymi} 

		Istnieje szereg aplikacji, które realizując swoje funkcjonalności korzystają z różnorodnych usług. W ten sposób mogą powstawać skomplikowane procesy obejmujące komunikację z wieloma komponentami programowymi jak i interakcje z użytkownikami. Dla tego typu zastosowań użyteczne jest wprowadzenie mechanizmów zarządzania procesami biznesowymi(ang. Business Process Management) - technologii zapewniającej kontrolę nad przebiegiem wieloetapowych procesów obejmujących różne usługi w środowisku wielo-domenowym.

		Narzędzia zarządzania procesami biznesowymi pozwalają na automatyzację procesów. Definicja procesów opiera się o schemat organizacji zadań(ang. workflow). Narzędzia BPM umożliwiają wizualizację, modelowanie i analizę procesów biznesowych. Zarządzanie procesami biznesowymi jest dzięki temu metodologią pozwalającą na tworzenie, przedstawianie i nadzorowanie procesów biznesowych oraz upraszcza zrozumienie przebiegu procesu. Mechanizmy BPM pozwalają na monitorowanie wykonania procesu biznesowego oraz dostęp do informacji o jego statusie. 

	\subsection{Zapewnienie bezpieczeństwa dostępu do usług w architekturze SOA}

		Zastosowanie architektury SOA wymusza nieco odmienne podejście do kwestii bezpieczeństwa niż w przypadku tradycyjnych aplikacji. SOA umożliwia kompozycję usług w usługi wyższego rzędu a także tworzenie z nich złożonych procesów biznesowych. Procesy takie mogą wykorzystywać serwisy należące do wielu organizacji i wielu domen bezpieczeństwa. Co za tym idzie, mechanizmy bezpieczeństwa wykorzystywane w modelu SOA powinny obejmować całość architektury, a nie pojedyncze serwisy.\cite{kanneganti2008soa}
		
		Mechanizm uwierzytelniania dla niektórych usług musi uwzględniać możliwość wywoływania serwisów przez użytkowników spoza domeny bezpieczeństwa w której jest uruchomiona dana usługa. Istnieje kilka podejść do uwierzytelniania w przypadku wywoływania usługi przez inny serwis. Jeżeli wymagania odnośnie uwierzytelniania w obu serwisach są identyczne, usługa powinna być w stanie wykorzystać uwierzytelnienie dokonane przez usługę wywołującą. W wypadku różnicy tych wymagań, może wystąpić konieczność ponownego uwierzytelnienia np. przy pomocy innych składników lub z wykorzystaniem innego źródła przechowującego dane użytkowników.

		Autoryzacja w środowisku SOA wykorzystującym kompozycję serwisów jest również bardziej złożona. Klient usługi wyższego rzędu musi być autoryzowany do wykonania wszystkich operacji w usługach składowych niezbędnych do zrealizowania jego żądania. Sprawdzenie tego powinno nastąpić w usłudze wyższego rzędu, w celu zapewnienia atomiczności serwisu w kontekście autoryzacji. W przeciwnym wypadku należy zapewnić możliwość wycofania niezrealizowanej z powodu jedynie częściowej autoryzacji transakcji z serwisów składowych.

		Model SOA zakłada intensywną wymianę komunikatów pomiędzy serwisami. Zapewnienie poufności transmisji danych w tradycyjnych systemach typu klient-serwer mogło być stosunkowo prosto zrealizowane poprzez szyfrowany kanał utworzony np. przy pomocy SSL(Secure Sockets Layer) lub TLS (Transport Layer Security). Rozwiązania związane z zapewnianiem bezpiecznego kanału transmisji danych zapewniają także integralność danych, co w połączeniu z gwarancją poufności przesyłanych danych zabezpiecza przed atakami typu man-in-the middle(TODO: zdefiniować tutaj albo gdzieś indziej). W przypadku SOA podejście to może okazać się niewystarczającym. Zabezpieczenia warstwy transportowej polegające na bezpiecznym kanale transmisji nie pozwalają na zapewnienie bezpieczeństwa w razie konieczności przetwarzania informacji przez węzły pośrednie. Nie umożliwiają również zapewnienia poufności jedynie fragmentu wiadomości a nie całej jej treści. Standardowe implementacje TLS nie gwarantują tożsamości nadawcy wiadomości(ang. non repudation). Rozwiązaniem tych problemów może być odpowiednie połączenie mechanizmów szyfrowania i podpisywania jedynie krytycznych z punktu widzenia bezpieczeństwa części wiadomości.
		
		Zapewnienie odpowiedniej ochrony prywatności danych w architekturze typu SOA może być trudniejsze niż w standardowej aplikacji np. z uwagi na możliwość istnienia serwisów wyższego rzędu zbudowanych z serwisów różnych podmiotów. Ze względu na to, że jest to aspekt nietechniczny, silnie powiązany z konkretnymi wymaganiami prawnymi i biznesowymi, postanowiliśmy nie uwzględniać go w naszych rozważaniach.

		Oprócz zagadnień bezpieczeństwa związanych bezpośrednio z SOA, istnieje również szereg innych kategorii potencjalnych zagrożeń wspólnych dla większości typów aplikacji sieciowych. Podatności na ataki można podzielić na trzy kategorie: związane z kodem aplikacji, z administracją lub z infrastrukturą. Z powodu obszerności tych zagadnień, nie mogą one zostać tutaj dokładnie opisane.

	\subsection{Wymagania niefunkcjonalne związane z bezpieczeństwem w architekturze SOA}	
	
		Implementując mechanizmy bezpieczeństwa w aplikacjach wykorzystujących SOA należy zwrócić szczególną uwagę na wymagania niefunkcjonalne wynikające z zastosowania tej architektury. Jedną z najważniejszych cech SOA jest zdolność do współpracy(ang. interoperability) pomiędzy serwisami. Różnica pomiędzy technologiami w których zostały zaimplementowane serwisy nie powinna stanowić przeszkody w integracji. Tak samo serwisy dostarczane przez różne podmioty muszą być w stanie współpracować ze sobą. Wymusza to stosowanie standardów definiujących strukturę komunikatów i protokołów określających reguły komunikacji pomiędzy serwisami. Co za tym idzie, bezpieczeństwo w SOA powinno być zapewniane przez standardowe mechanizmy.
		
		Istotnymi aspektami niefunkcjonalnym są także łatwość zarządzania bezpieczeństwem oraz łatwość implementacji i rozwoju danego rozwiązania. Aplikacje zgodne z paradygmatem SOA powinny być łatwo konfigurowalne i pozwalać na łatwe i szybkie dostosowanie ich do zmiennych wymagań biznesowych. W ramach zarządzania bezpieczeństwem można wyróżnić zarządzanie użytkownikami, grupami i ich uprawieniami w poszczególnych usługach oraz zarządzanie kryptograficznymi aspektami bezpieczeństwa np. wybór algorytmów szyfrujących czy długość stosowanego klucza. 


%---------------------------------------------------------------------------

\autsection{Cloud computing}{Tomasz Wójcik}
\label{sec:cloudComputing}
	
	Z pojęciem SOA związany jest model przetwarzania w chmurze obliczeniowej (ang. cloud computing). Chmura obliczeniowa może być jednym ze środków implementacji architektury zorientowanej na usługi, a sam model silnie zachęca do dekompozycji monolitycznych funkcjonalności do mniejszych jednostek abstrakcji, np. serwisów, które mogą być ze sobą łączone. Przedsiębiorstwa które korzystają z SOA mogą wprowadzić chmurę obliczeniową znacznie łatwiej niż te korzystające z innych typów aplikacji.  
	
	Cloud computing nie jest konceptem wynikłym z nagłego przełomu technologicznego, a raczej efektem połączenia kilku metodyk i technologii, z których część ma długą historię. Z uwagi na to, a także na ogromną popularność jaką termin ten zdobył w środowisku IT i związane z tym przemianowywanie rozmaitych usług hostingowych na terminologię związaną z cloudem, jest on pojęciem trudnym do zdefiniowania. Książka \cite{Rhoton09} przywołuje pracę \cite{Vaquero}, której głównym celem było stworzenie kompletnej definicji chmury obliczeniowej, na podstawie głównych cech przypisywanych temu konceptowi w literaturze. Wg. wspomnianej pracy, chmura obliczeniowa to „Duża pula łatwych w użyciu i łatwo dostępnych zwirtualizowanych zasobów, takich jak sprzęt, platformy do rozwoju oprogramowania, serwisy. Zasoby te mogą być dynamicznie przekonfigurowywane w celu dostosowania do różnego obciążenia(skalowanie), pozwalając na optymalne zużycie zasobów. W typowym modelu płatność jest uzależniona od stopnia zużycia puli zasobów, a poziom usług jest gwarantowany przez dostawcę infrastruktury w umowie typu Service Level Agreement”.

	Sama nazwa koncepcji wywodzi się z symbolu chmury, którym na diagramach sieciowych są oznaczane zewnętrzne  sieci. Początkowo za chmury obliczeniowe uważano tylko rozwiązania udostępniające zasoby poprzez internet, potem definicja uległa poszerzeniu. Zastosowanie chmury obliczeniowej umożliwia twórcom oprogramowania dostęp do najnowszych technologii i dynamiczne skalowanie aplikacji przy redukcji kosztów dzięki wykorzystaniu efektu skali przez dostawcę usługi.
 
	W zależności od poziomu abstrakcji dostarczanych usług możemy wyróżnić trzy poziomy chmury obliczeniowej, zwane także modelami usługi:
	
	\begin{itemize}
	\item Infrastructure as a service – najprostszy model chmury, dostarczający fizycznych, zazwyczaj zwirtualizowanych, zasobów sprzętowych wraz z systemem operacyjnym i niekiedy preinstalowanymi komponentami do uruchomienia aplikacji(serwery HTTP/Java EE, bazy danych)

	\item Platform as a service – model dostarczający użytkownikowi infrastruktury wraz z platformą do tworzenia i wdrażania oprogramowania w środowisku chmury obliczeniowej

	\item Software as a service – model dostarczający gotowe oprogramowanie użytkowe z którego mogą  bezpośrednio korzystać użytkownicy końcowi
	\end{itemize}
	
	Niezależnie od powyższego podziału chmury obliczeniowe można podzielić także ze względu podmiot będący dostawcą chmury. Podział ten wg. niektórych źródeł zwany także podziałem ze względu na model wdrożenia. Jeżeli chmura jest zarządzana przez wydzielony podmiot w obrębie tej samej organizacji która tworzy system, mówimy o chmurze prywatnej. Takie rozwiązanie posiada tylko nieliczne cechy przypisywane chmurze obliczeniowej, przede wszystkim te związane z wirtualizacją, a w ograniczonym stopniu także ze skalowalnością, która dotyczy tu tylko poszczególnych aplikacji a nie całości systemu. Co za tym idzie nie ma pełnej zgody czy chmurę prywatną można  uznać za pełnoprawną chmurę czy raczej za rozszerzenie konceptu wydzielonego centrum danych.  Chmura dedykowana dla konkretnego klienta może być równie dobrze zarządzana przez zewnętrzną firmę wyspecjalizowaną w dostarczaniu usług IT. Model ten jest określany jako partner cloud. W modelu community cloud, chmura jest dzielona przez grupę organizacji, najczęściej o zbliżonych wymaganiach dotyczących pewnych aspektów chmury obliczeniowej. Kontrolę nad nią może sprawować jedna z tych organizacji lub zewnętrzny podmiot. Ostatnim modelem jest chmura publiczna, zarządzana przez wyspecjalizowaną w tym rodzaju usług organizację i udostępniana dowolnej liczbie klientów poprzez infrastrukturę internetową. 

	W praktyce często spotykane są rozwiązania hybrydowe, będące połączeniem powyżej opisanych modeli wdrożenia. Szczególnie popularnym mechanizmem jest cloudbursting, czyli uruchomienie systemu w chmurze prywatnej umożliwiające skalowanie przy użyciu chmury publicznej w momencie nagłego wzrostu zapotrzebowania zasobów.

		
	\subsection{Bezpieczeństwo aplikacji w modelu chmury obliczeniowej}
	
	Przeniesienie aplikacji do chmury obliczeniowej, a szczególnie chmury dostarczanej i zarządzanej przez zewnętrzny podmiot, w lokalizacji nie będącej pod kontrolą klienta, wiąże się ze zmianami w modelu bezpieczeństwa.  Bezpieczeństwo całości systemu staje się odpowiedzialnością zarówno dostawcy usług cloudowych jak i ich klientów, tj. twórców oprogramowania. Dokładny podział odpowiedzialności zależy głównie od przyjętego modelu usługi i modelu wdrożenia\cite{PCI13}, ale należy mieć świadomość, że pomiędzy poszczególnymi rozwiązaniami w obrębie tych samych modeli mogą wystąpić istotne różnice. Wymusza to uzgodnienie wspólnych polityk i procedur bezpieczeństwa przez dostawcę i klienta, definiujących zakres odpowiedzialności tych podmiotów. 

	Zapewnienie bezpieczeństwa aplikacji działających w chmurze obliczeniowej, a więc potencjalnie wrogim środowisku, jest trudniejsze niż w środowisku nad którym mamy całkowitą kontrolę. Jednoczesne wykorzystywanie infrastruktury przez różnych klientów(multitenancy) i wirtualizacja wprowadzają nowe wektory ataku. 

%---------------------------------------------------------------------------

\autsection{Wymagania stawiane systemom Business-to-Business}{Krzysztof Wilaszek}
\label{sec:wymaganiaB2B}

Punktem wyjścia dla koncepcji systemów typu \textit{Business-to-Business} jest pojęcie systemów otwartych. Systemy otwarte można zdefiniować jako systemy spełniające warunki przenośności pomiędzy platformami sprzętowymi, skalowalności oraz zdolności do współpracy z innymi systemami\cite{Kajan04}. Systemu tego typu składają się z komponentów spełniających oczekiwane standardy. 

Rozwój sieci internet przyczynił się do powstania nowych form działalności biznesowej. Jedną z nich jest handel elektroniczny(ang. \textit{electronic commerce}) - proces kupowania, sprzedawania lub wymiany danych, usług i produktów poprzez internet. Jedną z form biznesowej działalności elektronicznej są systemy typu \textit{Business-to-Business}(B2B) - czyli systemy obsługujące transakcje pomiędzy partnerami biznesowymi. Systemy tego typu umożliwiają interakcje pomiędzy wieloma organizacjami w celu realizacji zadań zlecanych przez klientów aplikacji. Systemy typu B2B integrują różnorodne aplikacje dostarczając przy ich użyciu nowych funkcjonalności. 

Jednym z założeń systemów typu B2B jest dostępność danych i zasobów potrzebnych dla procesów realizowanych w ramach systemu. Komunikacje pomiędzy systemami oparta jest o model integracji \textit{Application-to-Application} umożliwiający wymianę informacji biznesowych pomiędzy współpracującymi organizacjami. Systemy typu B2B powinny oferować otwartość na rozszerzenia oraz skalowalność podczas dołączania nowych komponentów systemu. Powinny być oparte o proste komponenty pozwalające na ich wielokrotne użycie. Powinny dostarczać rozwiązań przenośnych pomiędzy różnymi platformami programowymi. Powinny również wspierać różne standardy zapewniania bezpieczeństwa dostępu do funkcjonalności systemu komputerowego. Komunikacja pomiędzy komponentami systemów typu \textit{Business-to-Business} może odbywać się np. za pośrednictwem serwisów webowych. Standaryzacji interakcji pomiędzy komponentami systemu dostarczać mogą narzędzia modelowania procesów biznesowym. Założenie różnorodności technologii wykorzystywanych przez komponenty systemu wymusza wdrożenie narzędzi translacji pomiędzy stosowanymi formatami wiadomości.

%---------------------------------------------------------------------------

\autsection{Zakres wymagań tworzonego systemu}{Krzysztof Wilaszek}
\label{sec:zakresWymagan}

	W ramach projektu implementacyjnego  związanego z niniejszą pracą przygotowano przykłady ilustrujące zastosowanie mechanizmów systemów zarządzania tożsamościami. Punktem wyjścia była implementacja mechanizmu jednokrotnego uwierzytelniania oraz jednokrotnego wylogowywania dla aplikacji z interfejsem udostępnianym poprzez przeglądarkę internetową. Głównym celem projektu jest analiza mechanizmów zarządzania tożsamościami dla architektury zorientowanej na usługi.

	Przygotowywane moduły powinny korzystać z narzędzi ograniczania dostępu do zasobów aplikacji dostarczanych przez serwer aplikacyjny. Mechanizm uwierzytelniania powinien być oparty o protokół LDAP. Jako specyfikację realizującą założenia systemów zarządzania tożsamościami wybrano standard SAML opisany w dalszej części pracy.

	Najważniejszą częścią pracy było zastosowanie koncepcji systemów zarządzania tożsamościami w architekturze SOA. Praca powinna przedstawiać propozycję rozwiązania problemów uwierzytelniania i autoryzacji dostępu do usług sieciowych. W ramach projektu wymagana jest implementacja modułów wykorzystujących uwierzytelnianie oparte o mechanizmy systemów zarządzania tożsamościami dla różnych standardów dostarczania usług webowych(np. SOAP i REST). Należy również dokonać analizy zastosowania zaproponowanych mechanizmów uwierzytelniania w architekturze zorientowanej na usługi. Powinien zostać opracowany mechanizm magistrali usług pozwalający na przekazywanie informacji uwierzytelniających pomiędzy modułami systemu. Wykorzystanie zaimplementowanych usług wraz z mechanizmami zabezpieczania dostępu powinno być przedstawione w postaci modelu procesu biznesowego. 

	Dla ilustracji opisywanych mechanizmów opracowano zestaw usług dostarczających prostych funkcjonalności realizujących różne etapy dokonywania zamówienia w sklepie internetowym(usługi sprawdzania stanu magazynu, zlecenia wydania towaru,   zlecenia transportu, rejestracji transakcji w systemie finansowym). Usługi korzystają z uwierzytelniania w systemie zarządzania tożsamościami. Usługi zaimplementowane zostały przy użyciu różnych technologii.Unifikację sposobu korzystania z serwisów osiągnięto dzięki modułowi magistrali usług. Używając zaimplementowanych usług opracowano model procesu biznesowego realizujący kompletną funkcjonalność dokonywania zamówienia.
