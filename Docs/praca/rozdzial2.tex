\chapter{Analiza wymagań}
\label{cha:analizaWymagan}

%---------------------------------------------------------------------------

\section{Koncepcja cyfrowego zarzadzania tozsamosciami(IdM)}
\label{sec:konceptcjaIdM}

Zarządzanie tożsamościami(Identity Management) jest podejściem do zagadnienia zapewnienia bezpieczeństwa dostępu do aplikacji w oparciu o dane identyfikujące użytkownika(Identity). Pojęcie danych identyfikujących może być zdefiniowane jako [3] (ITU-T Y.2720 Recommendation - numeracja z książki o IdM) "informacje o jednostce pozwalające na identyfikację tej jednostki dla pewnej dziedziny zastosowań". W kontekście systemów zarządzania cyfrowymi tożsamościami dane identyfikujące mogą dotyczyć nie tylko osób ale również na przykład komponentów programowych[książka o IdM]. Zgodnie z rekomendacją Y.2720 w skład danych identyfikujących wchodzą identyfikator jednostki, dane uwierzytelniające jednostkę oraz atrybuty opisujące jednostkę.
Elisa Bertino i Kenji Takahashi w książce "Identity Management" definiują cele zarządzania tożsamościami jako "utrzymanie integralności danych identyfikujących w trakcie ich użytkowania w celu udostępniania tych danych i powiązanych z nimi informacji w sposób bezpieczny i chroniący prywatność użytkowników". 

Role w koncepcji systemów zarządzania tożsamościami

Systemy zarządzania tożsamościami charakteryzują się rozdzieleniem odpowiedzialności związanych z dostarczaniem funkcjonalności oraz zadań związanych z zapewnieniem bezpieczeństwa dostępu do aplikacji. Usługi uwierzytelniania i autoryzacji oraz funkcjonalności systemów dostarczane są dla jednostek - składowych systemu zarządzania tożsamościami - zazwyczaj użytkowników.

Dane identyfikacyjne jednostek są gromadzone i wykorzystywane w trakcie korzystania z aplikacji. Dane osobowe mogą obejmować informacje związane z dokumentami tożsamości(numery dowodu osobistego, paszportu), dane bankowe, biometryczne oraz informacje związane z przebiegiem interakcji jednostki z systemem. Dane identyfikacyjne jednostek powinny być przetwarzane i gromadzone w sposób zapewniający  ochronę przed niewłaściwym użyciem. Nadużycia danych osobowych jednostek mogą prowadzić do istotnych strat użytkowników aplikacji.

Elementem przeprowadzającym weryfikację tożsamości jednostek jest usługa "Identity Provider". Usługa ta odpowiedzialna jest za przyporządkowywanie jednostkom atrybutów opisujących tożsamość, tworzenie powiązań pomiędzy różnymi atrybutami jednostki oraz tworzenie asercji zawierających informacje o atrybutach jednostek. 

%---------------------------------------------------------------------------

\section{Service-Oriented Architecture}
\label{sec:soa}

Service-Oriented Architecture

%---------------------------------------------------------------------------

\section{Wymagania stawiane systemom Business-to-Business}
\label{sec:wymaganiaB2B}

Wymagania stawiane systemom Business-to-Business

%---------------------------------------------------------------------------

\section{Zakres wymagan tworzonego systemu}
\label{sec:zakresWymagan}

Zakres wymagan tworzonego systemu

