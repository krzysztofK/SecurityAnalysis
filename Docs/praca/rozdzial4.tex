\chapter{Stos technologiczny}
\label{cha:stosTechnologiczny}

%---------------------------------------------------------------------------

\section{Podstawowe aspekty zwiazane z bezpieczenstwem systemów informatycznych}
\label{sec:aspektyBezpieczenstwa}

Podstawowe aspekty zwiazane z bezpieczenstwem systemów informatycznych

%---------------------------------------------------------------------------

\section{Mechanizmy bezpieczenstwa platformy Java SE}
\label{sec:javaSE}

Mechanizmy bezpieczenstwa platformy Java SE

%---------------------------------------------------------------------------


\section{Mechanizmy bezpieczenstwa platformy Java EE}
\label{sec:javaEE}

Mechanizmy bezpieczenstwa platformy Java EE

%---------------------------------------------------------------------------

\section{Mechanizmy bezpieczeństwa serwera aplikacyjnego JBoss}
\label{sec:jboss}

	Mechanizmy bezpieczeństwa serwera aplikacyjnego JBoss w wersji 7 oparte są o framework PicketBox. PicketBox dostarcza podstawowych funkcjonalności zapewnienia bezpieczeństwa dostępu do zasobów, takich jak uwierzytelnianie, autoryzacja, audyty systemu oraz mapowanie ról i danych uwierzytelniających. Usługi bezpieczeństwa dostarczane przez Picketbox są dostępne dla serwera aplikacyjnego poprzez podsystem bezpieczeństwa(ang. Security Subsystem).  Każdemu żądaniu klienta przypisywany jest kontekst bezpieczeństwa dostępny dla podsystemu bezpieczeństwa. 

	Kontekst bezpieczeństwa udostępnia komponenty skonfigurowane dla domeny bezpieczeństwa. Możliwe komponenty to:

	\begin{itemize}
		\item Authentication Manager - dokonuje uwierzytelniania użytkowników na podstawie otrzymanych danych uwierzytelniających przy użyciu modułów logowania zdefiniowanych dla wykorzystywanej domeny bezpieczeństwa;
		\item Authorization Manager - dostarcza informacji o rolach przypisanych użytkownikowi oraz dokonuje autoryzacji dostępu do zasobów dla uwierzytelnionych użytkowników;
		\item Audit Manager - pozwala na logowanie zdarzeń zachodzących w systemie zapewnienia bezpeiczeństwa dostępu do aplikacji;
		\item Mapping Manager - pozwala na przypisywanie uwierzytelnionemu użytkownikowi dodatkowych uprawnień, ról lub atrybutów.
	\end{itemize}

	\subsection{Konfiguracja podsystemy bezpieczeństwa}

		Korzystania z podsystemów bezpieczeństwa możliwe jest dzięki dodaniu rozszerzenia:
		\lstset{language=XML}
		\begin{lstlisting}
	<extension module="org.jboss.as.security"/>
		\end{lstlisting}
		w pliku konfiguracyjnym serwera.

		Podsystem bezpieczeństwa serwera aplikacyjnego JBoss udpstępnia konfigurację następujących własności:

		\begin{itemize}
			\item security-management - pozwala nadpisywać domyślne parametry modułu PicketBox takie jak implementacje klas zarządców dla procesów uwierzytelniania, autoryzacji, audytów, mapowania danych tożsamości oraz tworzenia relacji zaufania.
			\item security-domains - pozwala na konfiguracje domen bezpieczeństwa
			\item security-properties - pozwala definiować dodatkowe własności wymagane przez podsystem bezpieczeństwa.
		\end{itemize}

	\subsection{Konfiguracja domeny bezpieczeństwa}

		W ramach podsystemu bezpieczeństwa możliwa jest definicja domen bezpieczeństwa. Domeny bezpieczeństwa opisuje mechanizmy uwierzytelniania i autoryzacji ....

%---------------------------------------------------------------------------

\section{Security Assertion Markup Language}
\label{sec:saml}

Security Assertion Markup Language

%---------------------------------------------------------------------------

\section{Picketlink}
\label{sec:picketlink}

Picketlink

%---------------------------------------------------------------------------

\section{Business Process Management}
\label{sec:bpm}

Business Process Management

%---------------------------------------------------------------------------

\section{Enterprise Service Bus}
\label{sec:esb}

Enterprise Service Bus

%---------------------------------------------------------------------------

\section{OpenShift}
\label{sec:openShift}

OpenShift

%---------------------------------------------------------------------------