\chapter{Przeglad dostepnych rozwiazan}
\label{cha:przegladRozwiazan}

%---------------------------------------------------------------------------

\section{Najpopularniejsze standardy IdM}
\label{sec:standardyIdM}

Implementacja systemów zarządzania tożsamościami jest obszarem, w którym standaryzacja procesów wykorzystywania danych osobowych przynosi istotne korzyści. Wprowadzenie standardowych rozwiązań upraszcza wdrożenie nowych aplikacji lub usług typu "Identity Provider". Ujednoliceniu sposobu korzystania z funkcjonalności systemów zarządzania tożsamościami umożliwia tworzenie aplikacji klienckich używających podobnych rozwiązań dla dostępu do różnych usług. Rozwijanych jest wiele standardów realizujących wymagania stawiane systemom zarządzania tożsamościami. Najbardziej istotne z nich to SAML(Security Assertion Markup Language) oraz OpenID. 

\subsection{Security Assertion Markup Language}

SAML to oparty na języku XML standard zarządzania tożsamościami oraz wymiany informacji uwierzytelniających [3.1https://www.oasis-open.org/committees/download.php/13525/sstc-saml-exec-overview-2.0-cd-01-2col.pdf]. SAML oparty jest na podejściu wykorzystującym federację tożsamości - umożliwia tworzenie powiązań pomiędzy różnymi cyfrowymi tożsamościami użytkownika. SAML pozwala tworzyć asercje opisujące atrybuty tożsamości jednostki oraz przekazywać je do usług wymagających informacji identyfikujących swoich klientów.

\subsubsection{Cele technologii SAML}

Dokument [3.1] wymianie następujące cele stawiane technologii SAML:

\begin{itemize}
  \item niezależność od platformy - mechanizmy bezpieczeństwa powinny być niezależne od środowiska i implementacji usługi.
  \item luźne powiązanie pomiędzy elementami wchodzącymi w skład infrastruktury opartej o wymianę komunikatów SAML
  \item uproszczenie procesu uwierzytelniania z perspektywy klienta, np. poprzez wprowadzenie procedury SSO
\end{itemize}


https://www.oasis-open.org/committees/download.php/13525/sstc-saml-exec-overview-2.0-cd-01-2col.pdf

\subsection{OpenID}

%---------------------------------------------------------------------------

\section{Przyklady frameworków zarzadzajacych autentykacja i autoryzacja uzytkowników}
\label{sec:frameworki}

Przyklady frameworków zarzadzajacych autentykacja i autoryzacja uzytkowników

%---------------------------------------------------------------------------
