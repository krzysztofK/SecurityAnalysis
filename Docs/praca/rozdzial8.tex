\chapter{Wnioski}
\label{cha:wnioski}

Niniejsza praca przedstawia podejście do problemu zapewnienia bezpieczeństwa aplikacji w architekturze zorientowanej na usługi przy użyciu koncepcji systemów zarządzania tożsamościami. W ramach związanych z nią prac badawczych zaprojektowano i zaimplementowano system stanowiący przykład wykorzystania mechanizmów zarządzania tożsamościami dla aplikacji w architekturze SOA. Jako standard realizujący założenia systemów zarządzania tożsamościami proponowany system wykorzystuje specyfikację SAML. 

Przykładowy system prezentowany w pracy dowodzi użyteczności zastosowania koncepcji systemów typu \textit{IdM} w kontekście aplikacji opartych o architekturę SOA. Wykorzystanie asercji SAML umożliwiło implementację mechanizmów uwierzytelniania klientów serwisów webowych, dostarczanych przy użyciu różnych standardów udostępniania usług sieciowych(SOAP, REST). Zastosowanie protokołu SAML pozwoliło na implementację mechanizmów bezpieczeństwa systemu wykorzystującego różnorodne usługi w celu dostarczania wymaganych funkcjonalności. 

Wykorzystanie tokenów bezpieczeństwa bazujących na asercjach SAML jest rozwiązaniem dobrze dopasowanym do specyfiki aplikacji w architekturze SOA oraz do potrzeb związanych z wdrażaniem aplikacji na platformach chmur obliczeniowych. Jest podejściem realizującym założenia mechanizmów bezpieczeństwa na poziomie przesyłanych wiadomości. Dzięki temu pozwala na stosunkowo prostą implementację elementów charakterystycznych dla architektury typu SOA, w tym magistrali usług(ESB). Jest rozwiązaniem elastycznym; umożliwiającym efektywne wdrożenie dla różnorodnych dostawców i odbiorców usług oraz elementów pośrednich. Implementacja mechanizmów bezpieczeństwa w oparciu o protokół SAML umożliwia realizację procedury jednokrotnego uwierzytelniania klienta procesu biznesowego wykorzystującego niezależne od siebie usługi.

Zastosowanie mechanizmów bezpieczeństwa opartych o koncepcję zarządzania tożsamościami upraszcza rozwiązania części problemów, obserwowanych dla innych rozwiązań zapewniania bezpieczeństwa aplikacji w architekturze SOA. Jest podejściem bardziej efektywnym w procesie wdrażania dla poszczególnych elementów w rozbudowanej infrastrukturze systemu o architekturze SOA. Jest rozwiązaniem skalowalnym; charakteryzującym się łatwością rozszerzania - również poza granice domen wyznaczonych przez usługi uwierzytelniające.
%---------------------------------------------------------------------------