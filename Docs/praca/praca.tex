\documentclass[pdflatex,11pt]{aghdpl}
% \documentclass{aghdpl}               % przy kompilacji programem latex
% \documentclass[pdflatex,en]{aghdpl}  % praca w języku angielskim
\usepackage[polish]{babel}
\usepackage[utf8]{inputenc}

% dodatkowe pakiety
\usepackage{enumerate}
\usepackage{listings}
\lstloadlanguages{TeX}

\lstset{
  literate={ą}{{\k{a}}}1
           {ć}{{\'c}}1
           {ę}{{\k{e}}}1
           {ó}{{\'o}}1
           {ń}{{\'n}}1
           {ł}{{\l{}}}1
           {ś}{{\'s}}1
           {ź}{{\'z}}1
           {ż}{{\.z}}1
           {Ą}{{\k{A}}}1
           {Ć}{{\'C}}1
           {Ę}{{\k{E}}}1
           {Ó}{{\'O}}1
           {Ń}{{\'N}}1
           {Ł}{{\L{}}}1
           {Ś}{{\'S}}1
           {Ź}{{\'Z}}1
           {Ż}{{\.Z}}1
}

%---------------------------------------------------------------------------

\author{Krzysztof Wilaszek~Tomasz Wójcik}
\shortauthor{K. Wilaszek~T. Wójcik}
%\author{Tomasz Wójcik}
%\shortauthor{T. Wójcik}

\titlePL{Analiza narzędzi dla zarządzania mechanizmami bezpieczeństwa w aplikacjach o charakterze SOA w środowiskach wielodomenowych}
\titleEN{Tools analysis for security management of SOA applications in multi-domain environements}

\shorttitlePL{Analiza narzędzi dla zarządzania mechanizmami bezpieczeństwa w aplikacjach o charakterze SOA w środowiskach wielodomenowych} % skrócona wersja tytułu jeśli jest bardzo długi
\shorttitleEN{Tools analysis for security management of SOA applications in multi-domain environements}

\thesistypePL{Praca magisterska}
\thesistypeEN{Master of Science Thesis}

\supervisorPL{dr inż. Marcin Jarząb}
\supervisorEN{Marcin Jarząb Ph.D}

\date{2013}

\departmentPL{Katedra Informatyki}
\departmentEN{Department of Computer Science}

\facultyPL{Wydział Informatyki, Elektroniki i Telekomunikacji}
\facultyEN{Faculty of Computer Science, Electronics and Telecommunications}

\acknowledgements{Serdecznie dziękuję \dots prowadzącemu pracownię projektową!}



\setlength{\cftsecnumwidth}{10mm}

%---------------------------------------------------------------------------

\begin{document}

\titlepages

\tableofcontents
\clearpage

\chapter{Wstęp}
\label{cha:wstep}

%---------------------------------------------------------------------------

\section{Motywacja}
\label{sec:motywacja}

Zagadnienia związane z zapewnieniem bezpieczeństwa dostępu do aplikacji stanowią jeden z istonych problemów projektowania i implementacji systemów o charakterze rozpropszonym. Systemy tego typu dostarczają często skomplikowanych funkcjonalności i udostępniają poufne dane lub informacje o wysokiej wartości, takie jak np. wiadomości o stanie zdrowia, dane urzędowe lub informacje bankowe. Wymusza to stosowanie technik ograniczania dostępu do danych i funkcjonalności systemów informatycznych.

Systemy rozproszone budowane są niejednokrotnie w oparciu o złożoną architekturę wielu komponentów i powiązań pomiędzy nimi. W środowiskach tego typu duże wyzwanie stanowi zarządzanie informacjami uwierzytelniającymi użytkownika systemu oraz przekazywanie tych informacji pomiędzy składowymi systemu. Często zachodzi potrzeba integracji pomiędzy różnymi aplikacjami. Dostarcza to kolejnych wyzwań w zakresie zapewnienia poprawnych interakcji pomiędzy systemami.

Jednym z głównych czynników wpływających na skuteczność wdrażanych mechanizmów bezpieczeństwa aplikacji jest sposób ich wykorzystywania przez użytkowników. Zarządzanie przez użytkownika własnymi danymi uwierzytelniającymi jest ważnym elementem wpływającym na efektywność systemu zabezpieczeń aplikacji. Jednocześnie konieczność zapamiętywania rosnącej liczby haseł dla różnych usług stwarza zagrożenie związane ze stosowaniem słabych lub podobnych haseł dla różnych usług użytkownika. Jedną z koncepcji rozwiązania tego problemu jest mechanizm jednokrotnego uwierzytelniania, dzięki któremu użytkownik podając jedno bardziej skomplikowane hasło otwiera sobie drogę do korzystania z wielu usług. Implementacja tej koncepji stawia kolejne wyzwania związane z wymianą informacji uwierzytelniających pomiędzy różnymi systemami i tworzeniem relacji zaufania pomiędzy  aplikacjami.

%---------------------------------------------------------------------------

\section{Zarys dziedziny problemu}
\label{sec:zarysDziedzinyProblemu}

Niniejsza praca dotyka problemu jednokrotnego uwierzytelniania dla korzystania z wielu aplikacji. Dla tego typu zastosowań analizowane jest użycie różnych technik i metodologii pozwalających na gromadzenie i wymianę informacji na temat użytkownika pomiędzy aplikacjami. W wyniku wymiany tych informacji powinno być możliwe wdrożenie koncepcji jednokrotnego uwierzytelniania użytkownika.

Najprostszym przypadkiem zastosowania koncepcji jednokrotnego uwierzytelniania jest implementacja mechanizmu jednokrotnego logowania i wylogowywania użytkownika w różnych aplikacjach udostępnianych w przeglądarce internetowej. Najbardziej interesującym przypadkiem z punktu widzenia tematu pracy jest uwierzytelnienie klienta usług webowych.

Głównym zadaniem pracy jest analiza zarządzania mechanizmami bezpieczeństwa w architekturze zorientowanej na usługi(SOA)....













\chapter{Analiza wymagań}
\label{cha:analizaWymagan}

%---------------------------------------------------------------------------

\section{Koncepcja cyfrowego zarządzania tożsamościami(IdM)}
\label{sec:konceptcjaIdM}

	Zarządzanie tożsamościami(Identity Management) jest podejściem do zagadnienia zapewnienia bezpieczeństwa dostępu do aplikacji w oparciu o dane identyfikujące użytkownika(Identity). Pojęcie danych identyfikujących może być zdefiniowane jako ,,informacje o jednostce pozwalające na identyfikację tej jednostki dla pewnej dziedziny zastosowań''\cite{Itu09}. W kontekście systemów zarządzania cyfrowymi tożsamościami dane identyfikujące mogą dotyczyć nie tylko osób ale również na przykład komponentów programowych\cite{Bertino11}. Zgodnie z rekomendacją Y.2720 w skład danych identyfikujących wchodzą identyfikator jednostki, dane uwierzytelniające jednostkę oraz atrybuty opisujące jednostkę\cite{Itu09}.

	Elisa Bertino i Kenji Takahashi w książce ,,Identity Management'' definiują cele zarządzania tożsamościami jako ,,utrzymanie integralności danych identyfikujących w trakcie ich użytkowania w celu udostępniania tych danych i powiązanych z nimi informacji w sposób bezpieczny i chroniący prywatność użytkowników''\cite{Bertino11}.
	 
	\subsection{Role w koncepcji systemów zarządzania tożsamościami}

		Systemy zarządzania tożsamościami charakteryzują się rozdzieleniem odpowiedzialności związanych z dostarczaniem funkcjonalności oraz zadań związanych z zapewnieniem bezpieczeństwa dostępu do aplikacji. Usługi uwierzytelniania i autoryzacji oraz funkcjonalności systemów dostarczane są dla jednostek - składowych systemu zarządzania tożsamościami - zazwyczaj użytkowników.
		Dane identyfikacyjne jednostek są gromadzone i wykorzystywane w trakcie korzystania z aplikacji. Dane osobowe mogą obejmować informacje związane z dokumentami tożsamości(numery dowodu osobistego, paszportu), dane bankowe, biometryczne oraz informacje związane z przebiegiem interakcji jednostki z systemem. Dane identyfikacyjne jednostek powinny być przetwarzane i gromadzone w sposób zapewniający  ochronę przed niewłaściwym użyciem. Nadużycia danych osobowych jednostek mogą prowadzić do istotnych strat użytkowników aplikacji.
		Elementem przeprowadzającym weryfikację tożsamości jednostek jest usługa ,,Identity Provider''(IdP). Usługa ta odpowiedzialna jest za przyporządkowywanie jednostkom atrybutów opisujących tożsamość, tworzenie powiązań pomiędzy różnymi atrybutami jednostki oraz tworzenie asercji zawierających informacje o atrybutach jednostek. 

		Usługa 'Identity Provider' może współpracować z innymi usługami tego typu dołączając do danych uwierzytelniających informacje udostępniane przez inne zaufane usługi IdP. Wykorzystanie danych uwierzytelniających dostarczanych przez inne usługi uwierzytelniające wymaga wprowadzenia procesu zapewnienia wiarygodności otrzymywanych danych. Proces ten może opierać się na przypisywaniu miary wiarygodności do atrybutów uwierzytelniających. Wyznaczenie wartości tej miary może być na przykład oparte o ocenę stopnia zaawansowania mechanizmów weryfikacji danych uwierzytelniających wykorzystywanych przez usługę dostarczająca dany atrybut tożsamości.

		Dostawcy usług wykorzystujący infrastrukturę systemów zarządzania tożsamościami przed zezwoleniem na dostęp do swoich zasobów zlecają usługom typu IdP przeprowadzenie procesu uwierzytelniania klienta na podstawie otrzymanych danych uwierzytelniających. Dostawcy usług powinni mieć możliwość deklaracji poziomu skuteczności zabezpieczeń wykorzystywanych w procesie autoryzacji dostępu do określonych zasobów. Poziom ten może być różny w zależności od rodzaju zasobu.

		Przedstawiona w książce ,,Identity Management Concepts, Technologies, and Systems'' terminologia wprowadza również pojęcie jednostki nadzorującej\cite{Bertino11}. Jest to najczęściej instytucja upoważniona prawnie do nadzoru nad procesami przetwarzania i przechowania danych osobowych lub wglądu  w informacje o charakterze poufnym.

	\subsection{Relacje miedzy rolami w systemach zarządzania tożsamościami}

		\begin{figure}[h]
			\centering
			\includegraphics[width=15cm]{img/idmRelations.jpg}
			\caption{Relacje miedzy rolami w systemach zarządzania tożsamościami}
			\label{Relacje miedzy rolami w systemach zarządzania tożsamościami}
		\end{figure}

		Często stosowanym modelem jest infrastruktura złożona z jednej usługi typu ,,Identity Provider'' oraz wielu usług funkcjonalnych opierających na niej swoje mechanizmy zabezpieczeń. Dzięki takiemu rozwiązaniu dostawcy usług mogą skoncentrować się na tworzeniu funkcjonalności stanowiących istotę aplikacji - odpowiedzialności związane z zapewnieniem bezpieczeństwa dostępu delegowane są do usługi IdP. Specjalizowana usługa uwierzytelniania może dostarczać bardziej zaawansowanych zabezpieczeń. Dzięki realizacji tego modelu użytkownicy nie muszą zarządzać wieloma danymi uwierzytelniającymi dla różnych usług - dostęp do wielu serwisów gwarantowany jest przy użyciu tych samych danych. Wprowadza to jednak zagrożenia związane z centralizacja dostępu do różnych usług.

	\subsection{Federated Identity Management}

		Najczęściej użytkownicy korzystają nie z jednej usługi lecz z szerokiej gamy różnych usług. Każda z usług korzysta z własnych danych uwierzytelniających. Podejście ,,Federated Identity Management'' umożliwia tworzenie powiązań pomiędzy tożsamościami użytkownika w ramach różnych usług dzięki czemu dane uwierzytelniające każdej z sfederowanych usług mogą być wykorzystane w procesie uwierzytelniania dowolnej z usług.

	\subsection{Cykl życia tożsamości}

		Jednym z głównych zadań systemów zarządzania tożsamościami jest kontrola nad cyklem życia tożsamości. Autorzy książki ,,Identity Management: Concepts, Technologies and Systems'' opisują 4 etapy cyklu życiu tożsamości: tworzenie, użytkowanie, aktualizacja oraz wycofanie z użycia\cite{Bertino11}.

		Proces tworzenia cyfrowej tożsamości składa się z kilku kroków. Pierwszym z nich może być weryfikacja przedstawionych atrybutów tożsamości, wymagająca udowodnia przez jednostkę prawdziwości wprowadzanych danych. Następnie tworzone są dane uwierzytelniające. Ostatnim krokiem jest utworzenie tożsamości na podstawie otrzymanych danych oraz nadanie jednostce identyfikatora.

		Utworzona tożsamość może być wykorzystywana w różnych celach, np. zapewnienia wiarygodnej komunikacji lub w procesie jednokrotnego uwierzytelniania(ang. Single Sign-On).

		Systemy zarządzania tożsamościami powinny obsługiwać zmiany atrybutów tożsamości. Powinny aktualizować informacje o danych jednostek po zmianach wysyłając powiadomienia do usług przechowujących te dane, np. ,,Identity Provider''. Identyfikatory jednostek nie powinny podlegać zmianom. Systemy IdM muszą również usuwać tożsamości jeśli nie są już aktualne.

\section{Jednokrotne uwierzytelnianie}

	Książka ,,Identity Management: Concepts, Technologies and Systems'' definiuje jednokrotne uwierzytelnianie(ang. Single Sign-On) jako proces uwierzytelniania, w którym jednostka może wykorzystać wynik pojedynczego uwierzytelniania dla uzyskania dostępu do wielu niezależnych usług z ochroną dostępu\cite{Bertino11}. 

	Podstawą funkcjonowania mechanizmów jednokrotnego uwierzytelniania jest nawiązanie relacji zaufania pomiędzy dostawcami usług oraz serwisami typu ,,Identity Provider''. Po uwierzytelnieniu użytkownika w ramach jednej z usług objętych mechanizmem SSO dostęp do innej nie wymaga uwierzytelniania - dane uwierzytelniające są mapowane na dane niezbędne do uwierzytelnienia względem innej usługi oraz generowane są informacje pozwalające na uzyskanie dostępu do serwisu. Usługi korzystające z mechanizmu jednokrotnego uwierzytelniania powinny otrzymywać informacje kontekstowe o przebiegu procesu uwierzytelniania takie jak: wykorzystywane metody uwierzytelniania oraz sposób ochrony danych uwierzytelniających. Informacje te pozwalają na ocenę stopnia wiarygodności przeprowadzonego procesu uwierzytelniania.

	\subsubsection{Architektura systemów jednokrotnego uwierzytelniania}

		Implementacja mechanizmu jednokrotnego uwierzytelniania może opierać się o różne architektury. Autorzy książki ,,Identity Management: Concepts, Technologies and Systems''\cite{Bertino11} wymieniają następujące typy architektur systemów jednokrotnego uwierzytelniania:

		\begin{itemize}
		  \item Architektura oparta o brokery - architektura składająca się z punktu centralnego(serwera) oraz jednostek przez niego uwierzytelnianych. Serwer przydziela użytkownikom tokeny uwierzytelniające, dzięki którym możliwy jest dostęp do aplikacji. Przykładem architektury tego typu jest protokół Kerberos. 
		  \item Architektura oparta o agenty - architektura, w której w dostępie do każdej aplikacji pośredniczy agent uwierzytelniania. Jego rolą jest translacja pomiędzy metodą uwierzytelniania zastosowaną przez klienta a mechanizmami obsługiwanymi przez aplikację.
		  \item ,,Reverse proxy-based architecture'' - architektura wprowadzająca usługę proxy pośredniczącą w dostępie do wszystkich aplikacji objętych procedurą SSO. Moduł proxy dokonuje filtrowania przychodzących komunikatów - nieuwierzytelnione żądania zostają przekierowane, np. do serwera uwierzytelniającego.
		\end{itemize}
		  
		Procedura jednokrotnego uwierzytelniania pozwala na wygodniejszy sposób korzystania z aplikacji - znosi konieczność wielokrotnego wprowadzania danych identyfikujących. Użytkownik nie musi też zapamiętywać wielu haseł dla różnych usług. Wiele prostych haseł może zostać zastąpionych jedną bardziej wiarygodną metodą uwierzytelniania(np. z wykorzystaniem bardziej skomplikowanego hasła). Wadą wprowadzenia procedury jednokrotnego uwierzytelniania jest jednak centralizacja punktu dostępu do różnych aplikacji - złamanie zabezpieczeń otworzy drogę do wielu usług użytkownika.

%---------------------------------------------------------------------------

\section{Service-Oriented Architecture}
\label{sec:soa}

	Architektura zorientowana na usługi(ang. Service-Oriented Architecture) to architektura systemów informatycznych oparta o strukturę luźno powiązanych, rozproszonych usług, które mogą być wielokrotnie wykorzystywane i łączone w celu realizacji wymagań stawianych aplikacji. Architektura zorientowana na usługi umożliwia integrację usług w złożone procesy\cite{Lawler08}. Usługi składające się na architekturę SOA udostępniają swoje  funkcjonalności w ramach zdefiniowanych interfejsów i są od siebie niezależne. Komunikacja pomiędzy serwisami odbywa się poprzez wywołania dostarczanych przez nie operacji\cite{Papazoglou07}. 

	Architektura SOA zaprojektowana została jako rozwiązanie wielu problemów pojawiających się w trakcie tworzenia systemów rozproszonych. Do problemów tych należą: integracja aplikacji, zarządzanie transakcjami, zapewnienia bezpieczeństwa wykonywanych operacji, różnorodność środowisk uruchamiania aplikacji\cite{Papazoglou07}.

	Jednym z głównych założeń architektury zorientowanej na usługi jest niezależność od technologii implementacji. Jest to możliwe dzięki wykorzystaniu standardowych interfejsów usług oraz metod komunikacji pomiędzy nimi. Usługi w architekturze SOA są autonomiczne, same przechowują swój stan. W architekturze SOA wszelkie funkcjonalności są udostępniane w postaci usług.

	\subsection{Enterprise Service Bus} 

		W istniejących środowiskach dostarczania usług występuje często duża niejednorodność technologii, protokołów komunikacji i modeli wykorzystywanych przez różne serwisy. Jednym z rozwiązań tego problemu jest wprowadzenie warstwy pośredniczącej, która implementuje logikę pozwalającą na integrację różnorodnych usług. Rozwiązanie to stanowi podstawę dla koncepcji magistrali usług(ang. Enterprise Service Bus). 

		\begin{figure}[h]
			\centering
			\includegraphics[width=15cm,height=8cm]{img/esb.jpg}
			\caption{Schemat funkcjonowania magistrali usług}
			\label{Schemat funkcjonowania magistrali uslug}
		\end{figure}

		Magistrala ESB dostarcza funkcjonalności rozproszonego przetwarzania oraz integracji usług opartych o standardowe mechanizmy. Funkcje transportu i transformacji wiadomości pozwalają na komunikację pomiędzy niejednorodnymi i rozproszonymi usługami. Magistrala usług może zapewniać mechanizmy bezpieczeństwa dostępu do aplikacji, wiarygodności dostarczania danych oraz audytu komunikacji. ESB powinno pozwalać na przekazywanie informacji kontekstowych np. dotyczących transakcji lub bezpieczeństwa dostępu do usług.

		Mechanizmy ESB uniezależniają klientów usług od fizycznych właściwości dostarczania usługi. Dzięki zastosowaniu magistrali usług możliwe jest dokonywanie zmian po stronie usługi lub zastąpienie usługi inną bez wpływu na aplikację kliencką. 

	\subsection{Zarządzanie procesami biznesowymi} 

		Istnieje szereg aplikacji, które realizując swoje funkcjonalności korzystają z różnorodnych usług. W ten sposób mogą powstawać skomplikowane procesy obejmujące komunikację z wieloma komponentami programowymi jak i interakcje z użytkownikami. Dla tego typu zastosowań użyteczne jest wprowadzenie mechanizmów zarządzania procesami biznesowymi(ang. Business Process Management) - technologii zapewniającej kontrolę nad przebiegiem wieloetapowych procesów obejmujących różne usługi w środowisku wielo-domenowym.

		Narzędzie zarządzania procesami biznesowymi pozwalają na automatyzację procesów. Definicja procesów opiera się o schemat organizacji zadań(ang. workflow). Narzędzie BPM umożliwiają wizualizację, modelowanie i analizę procesów biznesowych. Zarządzanie procesami biznesowymi jest dzięki temu metodologią pozwalającą na tworzenie, przedstawianie i nadzorowanie procesów biznesowych oraz upraszcza zrozumienie przebiegu procesu. Mechanizmy BPM pozwalają na monitorowania wykonania procesu biznesowego oraz dostęp do informacji o jego statusie. 

	\subsection{Zapewnienie bezpieczeństwa dostępu do usług w architekturze SOA}

		Ważnym aspektem dostarczania usług sieciowych jest zapewnienie mechanizmów bezpieczeństwa w procesie komunikacji. Mechanizmy bezpieczeństwa mogą być włączone w warstwie transportowej procesu dostarczania usług lub mogą być realizowane na poziomie wiadomości przesyłanej pomiędzy komunikującymi się jednostkami\cite{Szychowiak09}. 

		Zabezpieczenia warstwy transportowej polegają na zestawieniu szyfrowanej komunikacji pomiędzy klientem i dostawcą usług. Dzięki temu możliwe jest zapewnienie poufności wymiany wiadomości pomiędzy uczestnikami procesu komunikacji i integralności dostarczanych danych. Możliwa jest również implementacja mechanizmów uwierzytelniania w oparciu o szyfrowanie komunikatów\cite{Kolaczek09}.

		Istnieją sytuacje, w których zabezpieczenia warstwy transportowej nie są wystarczające.  Nie pozwalają one na zapewnienie bezpieczeństwa w razie konieczności przetwarzania informacji przez węzły pośrednie. Nie umożliwiają również zapewnienia poufności jedynie fragmentu wiadomości a nie całej jej treści. Rozwiązaniem tego problemów jest zastosowanie koncepcji mechanizmów bezpieczeństwa na poziomie przesyłanych wiadomości. Koncepcja ta zakłada oparcie mechanizmów bezpieczeństwa o informacje dołączane do wiadomości wymienianych pomiędzy klientem i dostawcą usługi. Przesyłane informacje mogą pozwalać na przeprowadzenie procesu uwierzytelniania lub zapewnienie integralności komunikatów. Istnieją liczne standardy opisujące format danych dołączanych do wiadomości oraz sposób ich przesyłania przy pomocy protokołów wykorzystywanych do dostarczania usług. Standardy tego typu definiują również sposób szyfrowania wymienianych wiadomości.		
%---------------------------------------------------------------------------

\section{Wymagania stawiane systemom Business-to-Business}
\label{sec:wymaganiaB2B}

Wymagania stawiane systemom Business-to-Business

%---------------------------------------------------------------------------

\section{Zakres wymagań tworzonego systemu}
\label{sec:zakresWymagan}

	W ramach projektu implementacyjnego  związanego z niniejszą pracą przygotowano przykłady ilustrujące zastosowanie mechanizmów systemów zarządzania tożsamościami. Punktem wyjścia była implementacja mechanizmu jednokrotnego uwierzytelniania oraz jednokrotnego wylogowywania dla aplikacji z interfejsem udostępnianym poprzez przeglądarkę internetową. Głównym celem projektu jest analiza mechanizmów zarządzania tożsamościami dla architektury zorientowanej na usługi.

	Przygotowywane moduły powinny korzystać z narzędzi uwierzytelniania i autoryzacji dostarczanych przez serwer aplikacyjny. Mechanizm uwierzytelniania powinien być oparty o protokół LDAP. Jako specyfikację realizującą założenia systemów zarządzania tożsamościami wybrano standard SAML opisany w dalszej części pracy.

	Najważniejszą częścią pracy było zastosowanie koncepcji systemów zarządzania tożsamościami w architekturze SOA. Praca powinna przedstawiać propozycję rozwiązania problemów uwierzytelniania i autoryzacji dostępu do usług sieciowych. W ramach projektu wymagana jest implementacja modułów wykorzystujących uwierzytelnianie oparte o mechanizmy systemów zarządzania tożsamościami dla różnych standardów dostarczania usług webowych(np. SOAP i REST). Należy również dokonać analizy zastosowania zaproponowanych mechanizmów uwierzytelniania w architekturze zorientowanej na usługi. Powinien zostać opracowany mechanizm magistrali usług pozwalający na przekazywanie informacji uwierzytelniających pomiędzy modułami systemu. Wykorzystanie zaimplementowanych usług wraz z mechanizmami zabezpieczania dostępu powinno być przedstawione w postaci modelu procesu biznesowego. 

	Dla ilustracji opisywanych mechanizmów opracowano zestaw usług dostarczających prostych funkcjonalności realizujących różne etapy dokonywania zamówienia w sklepie internetowym(usługi sprawdzania stanu magazynu, zlecenia wydania towaru,   zlecenia transportu, rejestracji transakcji w systemie finansowym). Usługi korzystają z uwierzytelniania w systemie zarządzania tożsamościami. Usługi zaimplementowane zostały przy użyciu różnych technologii.Unifikację sposobu korzystania z serwisów osiągnięto dzięki modułowi magistrali usług. Używając zaimplementowanych usług opracowano model procesu biznesowego realizujący kompletną funkcjonalność dokonywania zamówienia.

\chapter{Przeglad dostepnych rozwiazan}
\label{cha:przegladRozwiazan}

%---------------------------------------------------------------------------

\section{Najpopularniejsze standardy IdM}
\label{sec:standardyIdM}

Implementacja systemów zarządzania tożsamościami jest obszarem, w którym standaryzacja procesów wykorzystywania danych osobowych przynosi istotne korzyści. Wprowadzenie standardowych rozwiązań upraszcza wdrożenie nowych aplikacji lub usług typu "Identity Provider". Ujednoliceniu sposobu korzystania z funkcjonalności systemów zarządzania tożsamościami umożliwia tworzenie aplikacji klienckich używających podobnych rozwiązań dla dostępu do różnych usług. Rozwijanych jest wiele standardów realizujących wymagania stawiane systemom zarządzania tożsamościami. Najbardziej istotne z nich to SAML(Security Assertion Markup Language) oraz OpenID. 

\subsection{Security Assertion Markup Language}

	SAML to oparty na języku XML standard zarządzania tożsamościami oraz wymiany informacji uwierzytelniających [3.1https://www.oasis-open.org/committees/download.php/13525/sstc-saml-exec-overview-2.0-cd-01-2col.pdf]. SAML oparty jest na podejściu wykorzystującym federację tożsamości - umożliwia tworzenie powiązań pomiędzy różnymi cyfrowymi tożsamościami użytkownika. SAML pozwala tworzyć asercje opisujące atrybuty tożsamości jednostki oraz przekazywać je do usług wymagających informacji identyfikujących swoich klientów.

	\subsubsection{Cele technologii SAML}

		Dokument [3.1] wymianie następujące cele stawiane technologii SAML:

		\begin{itemize}
		  \item niezależność od platformy - mechanizmy bezpieczeństwa powinny być niezależne od środowiska i implementacji usługi.
		  \item luźne powiązanie pomiędzy elementami wchodzącymi w skład infrastruktury opartej o wymianę komunikatów SAML
		  \item uproszczenie procesu uwierzytelniania z perspektywy klienta, np. poprzez wprowadzenie procedury SSO
		  \item redukcja kosztów administracyjnych poprzez zastąpienie wielu oddzielnych modułów bezpieczeństwa jednym wspólnym dla  różnych aplikacji
		\end{itemize}

	\subsubsection{Struktura specyfikacji SAML}

		Specyfikacja technologii SAML definiuje czterowarstwową strukturę, w skład której wchodzą asercje, protokoły, mapowania dla protokołów komunikacyjnych oraz profile. 
		Asercje zawierają informacje wymieniane pomiędzy aplikacjami, usługami "Identity Provider" oraz użytkownikami. Protokoły, mapowania oraz profile definiują mechanizmy przetwarzania asercji.

		\paragraph{Asercje}\mbox{}\\
					
			Asercje są to wiadomości zawierające dane identyfikacyjne jednostek w systemie. Składają się z deklaracji tożsamości opisujących jednostki wygenerowanych przez usługę "Identity Provider" systemu SAML. Na podstawie otrzymanych deklaracji tożsamości jednostki dostawca usługi podejmuje decyzję o przyznaniu lub odmówieniu prawa dostępu do swoich zasobów. Również dostawcy usług mają możliwość tworzenia asercji w celu utworzenia zapytania do serwisu uwierzytelniającego o parametry transakcji określania tożsamości. 

			\subparagraph{Struktura asercji}\mbox{}\\
			
				\begin{figure}[h]
				\centering
					\includegraphics{img/samlAssertion.jpg}
				\caption{Elementy asercji SAML}
				\label{Elementy asercji SAML}
				\end{figure}

				Asercja SAML zawiera informacje o wystawiającym asercję. Może zawierać również informację o dacie wygenerowania asercji. W celu zapewnienia integralności informacji do asercji dołączany jest cyfrowy podpis wystawiającego. W dalszej części asercji znajdują się dane opisujące jednostkę, względem której utworzono asercję. Następną sekcją wiadomości są warunki, pod którymi asercja może być wykorzystywana. W tym fragmencie mogą znajdować się informacje o okresie ważności asercji lub usługi, do których adresowana jest asercja. Ostatnim elementem jest deklaracja tożsamości, zawierające informacje kontekstowe o procesie uwierzytelniania, np. dotyczące zastosowanej metody uwierzytelniania.

			\subparagraph{Typy deklaracji tożsamości}\mbox{}\\

				Dokument [31] definiuje następujące typy deklaracji tożsamości zawartych w asercjach SAML:

				\begin{itemize}
				  \item deklaracja uwierzytelniania - stwierdza, że opisana w asercji jednostka została uwierzytelniona w danym momencie przy użyciu mechanizmów opisanych w opisie kontekstu deklaracja
				  \item deklaracja atrybutu - stwierdza, że dany atrybut o zadanej wartości jest przypisany jednostce
				  \item deklaracja autoryzacji - stwierdza, że jednostce opisanej w asercji przyznano lub odmówiono praw dostępu do zasobów pod określonymi warunkami
				 \end{itemize}

		\paragraph{Protokoły}\mbox{}\\ 

			Protokoły SAML definiują format wiadomości żądań i odpowiedzi pozwalających na komunikację pomiędzy elementami systemu zarządzania tożsamościami przy pomocy technologii SAML. Specyfikacja SAML określa protokoły:

			\begin{itemize}
			  \item protokół odpytywania usługi "Identity Provider" o asercje
			  \item protokół żądania uwierzytelniania jednostki
			  \item protokół rejestrowania identyfikatorów jednostek
			  \item protokół żądania wygaśnięcia identyfikatora jednostki
			  \item protokół żądania jednokrotnego wylogowania z wielu aplikacji
			  \item protokół żądania odzwierciedlenia pomiędzy różnymi identyfikatorami jednostki
			\end{itemize}

		\paragraph{Mapowania dla protokołów komunikacyjnych}\mbox{}\\

			Mapowania SAML do protokołów komunikacyjnych określają w jaki sposób wiadomości protokołów SAML powinny być przekazywane przy pomocy standardowych protokołów komunikacyjnych.  Mapowania mogą np. definiować sposób przesyłania wiadomości SAML przy pomocy protokołów HTTP lub SOAP.

		\paragraph{Profile}\mbox{}\\

			Profile SAML definiują zbiór funkcjonalności jakie można uzyskać przy użyciu elementów niższych warstw(asercji, protokołów i mapowań) oraz sposób w jaki te funkcjonalności mogą być osiągnięte. Przykładem mogą być profile jednokrotnego uwierzytelniania specyfikujące sposób komunikacji pomiędzy dostawcami usług i serwisami "Identity Provider" w celu dostarczenia mechanizmów SSO lub profile zapytania o asercję dla jednostki.

	\subsubsection{Mechanizmy jednokrotnego uwierzytelniania przy użyciu SAML}

		\paragraph{Schemat funkcjonowania mechanizmów SSO w oparciu o technologię SAML}\mbox{}\\

			\begin{figure}[h]
				\centering
					\includegraphics[width=15cm,height=2.5cm]{img/samlSSO.jpg}
				\caption{Schemat funkcjonowania mechanizmu jednokrotnego uwierzytelniania w SAML}
				\label{Schemat funkcjonowania mechanizmu jednokrotnego uwierzytelniania w SAML}
			\end{figure}

			Aby istniała możliwość korzystania z mechanizmu jednokrotnego uwierzytelniania konieczne jest utworzenie federacji pomiędzy tożsamościami jednostki. Po utworzeniu powiązań pomiędzy różnymi tożsamościami użytkownik po poprawnym uwierzytelnieniu względem jednej z usług może korzystać z innych sfederowanych serwisów. Wylogowanie się wykonane dla którejś z aplikacji powoduje zamknięcie dostępu do wszystkich sfederowanych usług. Procedura jednokrotnego uwierzytelniania może być wykorzystywana do momentu usunięcia federacji pomiędzy tożsamościami użytkownika.

		\paragraph{Przebieg procesu jednokrotnego uwierzytelniania w SAML}\mbox{}\\

			\begin{figure}[h]
				\centering
					\includegraphics[width=15cm,height=10cm]{img/ssoSteps.jpg}
				\caption{Przebieg procesu jednokrotnego uwierzytelniania w SAML}
				\label{Przebieg procesu jednokrotnego uwierzytelniania w SAML}
			\end{figure}

			Usługa objęta mechanizmem jednokrotnego uwierzytelniania po otrzymaniu żądania udostępnienia swoich zasobów zleca modułowi "Identity Provider" przeprowadzenie procesu uwierzytelnienia klienta serwisu. Asercja będąca wynikiem procesu uwierzytelnienia zostaje przekazana do usługi zlecającej. Usługa dokonuje weryfikacji otrzymanej asercji i akceptuje lub odrzuca żądanie dostępu do zasobów. Gdy użytkownik chce skorzystać z usług innego serwisu sfederowanego z usługą, do której otrzymał dostęp, proces uwierzytelniania przebiega podobnie. Pomijany jest jednak krok ponownego uwierzytelniania użytkownika przez usługę "Identity Provider" - usługa ta zwraca do aplikacji żądającej uwierzytelniania asercję wygenerowaną na podstawie wcześniej wykonanego procesu uwierzytelniania.	

\subsection{OpenID}

\subsection{Liberty Identity Web Services Framework}

	Identity Web Services Framework(ID-WSF) [http://docs.oracle.com/cd/E19575-01/820-3746/ghfgd/index.html] to zbiór specyfikacji definiujących mechanizmy dla zapewnienia bezpieczeństwa serwisów webowych, pomiędzy którymi istnieją relacje zaufania(federacje). ID-WSF określa podejście do zarządzania tożsamościami, w którym tożsamości jednostek zarządzane są przez różnych dostawców usług połączonych relacją federacji. Usługi webowe mogą wymieniać między sobą dane osobowe swoich użytkowników w celu dostarczenia funkcjonalności żądanej przez klienta usługi. Przykładem sytuacji, gdzie może być zastosowany ten model jest usługa potrzebująca dostarczenia danych adresowych swojego użytkownika. W tym celu może skorzystać z zasobów innej usługi dysponującej tymi danymi.

	Specyfikacja ID-WSF wprowadza dodatkowy element do architektury systemów zarządzania tożsamościami - "Discovery Service". Usługa "Discovery Service" pozwala dostawcom usług na wyszukiwanie innych serwisów dostarczających funkcjonalności pozwalających na realizacje żądań zadanych usłudze.

	
%---------------------------------------------------------------------------

\section{Przyklady frameworków zarzadzajacych autentykacja i autoryzacja uzytkowników}
\label{sec:frameworki}

Przyklady frameworków zarzadzajacych autentykacja i autoryzacja uzytkowników

%---------------------------------------------------------------------------

\chapter{Stos technologiczny}
\label{cha:stosTechnologiczny}

%---------------------------------------------------------------------------

\section{Podstawowe aspekty zwiazane z bezpieczenstwem systemów informatycznych}
\label{sec:aspektyBezpieczenstwa}

Podstawowe aspekty zwiazane z bezpieczenstwem systemów informatycznych

%---------------------------------------------------------------------------

\section{Mechanizmy bezpieczenstwa platformy Java SE}
\label{sec:javaSE}

Mechanizmy bezpieczenstwa platformy Java SE

%---------------------------------------------------------------------------


\section{Mechanizmy bezpieczenstwa platformy Java EE}
\label{sec:javaEE}

Mechanizmy bezpieczenstwa platformy Java EE

%---------------------------------------------------------------------------

\section{Mechanizmy bezpieczeństwa serwera aplikacyjnego JBoss}
\label{sec:jboss}

	Mechanizmy bezpieczeństwa serwera aplikacyjnego JBoss w wersji 7 oparte są o framework PicketBox. PicketBox dostarcza podstawowych funkcjonalności zapewnienia bezpieczeństwa dostępu do zasobów, takich jak uwierzytelnianie, autoryzacja, audyty systemu oraz mapowanie ról i danych uwierzytelniających. Usługi bezpieczeństwa dostarczane przez Picketbox są dostępne dla serwera aplikacyjnego poprzez podsystem bezpieczeństwa(ang. Security Subsystem).  Każdemu żądaniu klienta przypisywany jest kontekst bezpieczeństwa dostępny dla podsystemu bezpieczeństwa. 

	Kontekst bezpieczeństwa udostępnia komponenty skonfigurowane dla domeny bezpieczeństwa. Możliwe komponenty to:

	\begin{itemize}
		\item Authentication Manager - dokonuje uwierzytelniania użytkowników na podstawie otrzymanych danych uwierzytelniających przy użyciu modułów logowania zdefiniowanych dla wykorzystywanej domeny bezpieczeństwa;
		\item Authorization Manager - dostarcza informacji o rolach przypisanych użytkownikowi oraz dokonuje autoryzacji dostępu do zasobów dla uwierzytelnionych użytkowników;
		\item Audit Manager - pozwala na logowanie zdarzeń zachodzących w systemie zapewnienia bezpeiczeństwa dostępu do aplikacji;
		\item Mapping Manager - pozwala na przypisywanie uwierzytelnionemu użytkownikowi dodatkowych uprawnień, ról lub atrybutów.
	\end{itemize}

	\subsection{Konfiguracja podsystemy bezpieczeństwa}

		Korzystania z podsystemów bezpieczeństwa możliwe jest dzięki dodaniu rozszerzenia:
		\lstset{language=XML}
		\begin{lstlisting}
	<extension module="org.jboss.as.security"/>
		\end{lstlisting}
		w pliku konfiguracyjnym serwera.

		Podsystem bezpieczeństwa serwera aplikacyjnego JBoss udpstępnia konfigurację następujących własności:

		\begin{itemize}
			\item security-management - pozwala nadpisywać domyślne parametry modułu PicketBox takie jak implementacje klas zarządców dla procesów uwierzytelniania, autoryzacji, audytów, mapowania danych tożsamości oraz tworzenia relacji zaufania.
			\item security-domains - pozwala na konfiguracje domen bezpieczeństwa
			\item security-properties - pozwala definiować dodatkowe własności wymagane przez podsystem bezpieczeństwa.
		\end{itemize}

	\subsection{Konfiguracja domeny bezpieczeństwa}

		W ramach podsystemu bezpieczeństwa możliwa jest definicja domen bezpieczeństwa. Domeny bezpieczeństwa opisuje mechanizmy uwierzytelniania i autoryzacji ....

%---------------------------------------------------------------------------

\section{Security Assertion Markup Language}
\label{sec:saml}

Security Assertion Markup Language

%---------------------------------------------------------------------------

\section{Picketlink}
\label{sec:picketlink}

Picketlink

%---------------------------------------------------------------------------

\section{Business Process Management}
\label{sec:bpm}

Business Process Management

%---------------------------------------------------------------------------

\section{Enterprise Service Bus}
\label{sec:esb}

Enterprise Service Bus

%---------------------------------------------------------------------------

\section{OpenShift}
\label{sec:openShift}

OpenShift

%---------------------------------------------------------------------------
\chapter{Stos technologiczny}
\label{cha:stosTechnologiczny}

{\it

W celu realizacji wymagań stawianych opracowywanym aplikacjom wykorzystane zostały różne technologie umożliwiające implementację określonych aspektów dostarczanych funkcjonalności.

Wdrożenie zabezpieczeń dostępu do aplikacji oparte zostało na mechanizmach bezpieczeństwa definiowanych przez specyfikację Java EE i dostarczanych przez serwer aplikacyjny JBoss. Serwer JBoss pozwala konfigurować systemy i domeny bezpieczeństwa oraz stosowane w ich obrębie mechanizmy uwierzytelniania i autoryzacji użytkowników usług. Proces uwierzytelniania klientów aplikacji wykorzystuje mechanizmy implementowane poprzez protokół LDAP. 

Przykładowe aplikacje wykorzystują realizację standardu SAML - projekt Picketlink. Picketlink definiuje model tożsamości wykorzystywany w procesie uwierzytelniania, pozwala na stosowanie standardu - WS-Trust w celu obsługi funkcjonalności federacji tożsamości. Dostarcza implementację modułów ,,Identity Provider'' oraz ,,Security Token Service''. 

}

%---------------------------------------------------------------------------
\autsection{Mechanizmy bezpieczeństwa platformy Java SE}{Krzysztof Wilaszek}
\label{sec:javaSE}

	Architektura platformy Java zaprojektowana została z uwzględnieniem różnorodnych aspektów bezpieczeństwa aplikacji komputerowych. Najbardziej podstawowe mechanizmy bezpieczeństwa dotyczą poziomu języka programowania - można wśród nich wymienić bezpieczeństwo typów, automatyczne zarządzanie pamięcią, mechanizm \textit{,,garbage collection''} czy też mechanizm bezpiecznego ładowania klas. Składnia języka Java umożliwia również definiowanie modyfikatorów dostępu do pól i metod klas. Kod napisany w języku Java kompilowany jest do niezależnego od platformy kodu bajtowego, który przed wykonaniem weryfikowany jest pod kątem składni, zarządzania pamięcią, operacji na stosie czy też bezpieczeństwa typów\cite{Oracle13}.

	Wraz z rozwojem platformy Java wprowadzano zmiany w architekturze bezpieczeństwa Javy wzbogacając funkcjonalności platformy o powszechnie wykorzystywane algorytmy, mechanizmy i protokoły. Należą do nich algorytmy kryptograficzne, interfejsy dla infrastruktury klucza publicznego, interfejsy uwierzytelniania i kontroli dostępu do zasobów aplikacji. Platforma Javy zapewnia niezależność aplikacji wykorzystujących API mechanizmów bezpieczeństwa od konkretnej, niejednokrotnie złożonej implementacji danego mechanizmu. Umożliwia również dostarczenie własnej realizacji dla różnych mechanizmów bezpieczeństwa.

	\subsection{Architektura platformy Java SE dla mechanizmów bezpieczeństwa}

		Podstawowymi założeniami architektury platformy Javy dla mechanizmów bezpieczeństwa są niezależność aplikacji wykorzystującej konkretne mechanizmy bezpieczeństwa od ich implementacji, zdolność do współpracy pomiędzy różnymi aplikacjami i serwisami bezpieczeństwa oraz rozszerzalność zbioru dostarczanych mechanizmów bezpieczeństwa. Podstawowym elementem architektury bezpieczeństwa platformy Java realizującym te założenia jest klasa \textit{java.security.Provider}. Klasa dostarcza implementacji określonych usług bezpieczeństwa. Dostawcom usług bezpieczeństwa przyporządkowane mogą być priorytety - w przypadku kilku implementacji tego samego serwisu wybierany jest ten od dostawcy z najwyższym priorytetem. 

		Implementacja platformy Java zawiera zbiór gotowych dostawców dla różnych usług bezpieczeństwa takich jak kryptografia, uwierzytelnianie, etc. Zgodnie z założeniami architektonicznymi zbiór funkcjonalności może być rozszerzany oraz wykorzystywany przez różnorodne współpracujące aplikacje. Usługi mogą być konfigurowane w celu ich dostosowania do potrzeb konkretnych aplikacji. 

	\subsection{Usługi bezpieczeństwa platformy Java SE}

		Platforma Java udostępnia szeroki wachlarz usług dostarczających funkcjonalności algorytmów kryptograficznych takich jak MDA, podpis cyfrowy, szyfrowanie symetryczne(blokowe i strumieniowe) oraz asymetryczne, szyfrowanie przy użyciu haseł, kryptografia krzywych eliptycznych, protokół uzgadniania kluczy, generator kluczy, kod uwierzytelniania wiadomości, generator liczb pseudolosowych. Dostarcza implementacji bezpiecznej wymiany wiadomości przy użyciu infrastruktury klucza publicznego(ang. \textit{Public Key Infrastructure}). Jest to podejście, w którym tożsamość klientów aplikacji potwierdzana jest przez certyfikaty cyfrowe generowane przez \textit{Certification Authorities}. Platforma Javy dostarcza implementacji wspierających certyfikaty X.509 oraz listy unieważnionych certyfikatów(ang. \textit{Certificate Revocation Lists}). Platforma dostarcza mechanizmów przechowywania kluczy i certyfikatów bezpieczeństwa w oparciu o różne standardy. Dostarcza API i narzędzi umożliwiających operacje na zgromadzonych danych.

		Platforma Javy dostarcza również API umożliwiające uwierzytelnianie klientów aplikacji przy użyciu modułów logowania. Architektura podsystemu uwierzytelniania umożliwia wprowadzanie nowych modułów logowania rozszerzających dostępne funkcjonalności uwierzytelniania. Domyślnie dostępne są metody uwierzytelniania oparte o protokoły LDAP i Kerberos oraz o dane zgromadzone w magazynach kluczy.

		W ramach platformy Java zaimplementowane zostały mechanizmy zapewnienia bezpieczeństwa komunikacji sieciowej - gwarantowania poufności komunikacji oraz zapewnienia integralności danych. Jednym z tych mechanizmów są protokoły SSL oraz TLS dostarczające funkcjonalności  szyfrowania przesyłanych danych, weryfikacji ich integralności oraz uwierzytelniania klienta i serwera. Platforma dostarcza również implementacji standardu SASL opisującego proces uwierzytelniania w komunikacji między klientem i serwerem. 

		Innym ważnym mechanizmem implementowanym w ramach usług bezpieczeństwa platformy Java jest mechanizm kontroli dostępu do zasobów aplikacji. Moduł kontroli dostępu oparty jest o managera bezpieczeństwa(ang. \textit{security manager})  podejmującego decyzję o przydzieleniu lub odmowie dostępu do określonych zasobów aplikacji. Decyzja ta podejmowana jest na podstawie informacji skojarzonych z kodem aplikacji na etapie jego ładowania przez \textit{class loadera}  takich jaki pochodzenie kodu, jednostka podpisująca kod oraz uprawnienia przyznane kodowi. W procesie podejmowania decyzji uwzględniane są polityki zdefiniowane w obrębie aplikacji. 

		Platforma dostarcza również implementacji standardu cyfrowego podpisu XML gwarantującego integralność danych oraz uwierzytelnienie wiadomości i jej nadawcy.
	
%---------------------------------------------------------------------------

\autsection{Mechanizmy bezpieczeństwa platformy Java EE}{Krzysztof Wilaszek}
\label{sec:javaEE}

	Specyfikacja Java EE definiuje mechanizmy bezpieczeństwa dla komponentów wchodzących w skład poszczególnych warstw aplikacji. Implementacja mechanizmów bezpieczeństwa dla komponentów dostarczana jest przez kontener aplikacji. Istnieją dwa rodzaje definiowania wymagań bezpieczeństwa aplikacji - deklarowalne i programowalne\cite{Oracle12}. Deklarowalna definicja bezpieczeństwa aplikacji jest to opis wymagań bezpieczeństwa zapisany przy użyciu adnotacji lub deskryptora aplikacji. Przy użyciu tej metody możliwe jest zdefiniowanie ról, ograniczeń dostępu do poszczególnych zasobów aplikacji oraz wymagań dotyczących procesu uwierzytelniania użytkowników. Drugie podejście - programowalna definicja bezpieczeństwa aplikacji jest opisem wymagań bezpieczeństwa zawartym w kodzie aplikacji, służącym podejmowaniu decyzji dotyczących bezpieczeństwa aplikacji na etapie realizacji jej funkcjonalności. Podejście to jest najczęściej wykorzystywane gdy metoda oparta o deklaracje jest niewystarczająca w danej sytuacji.

	Na aplikację Java EE mogą składać się zasoby chronione i niechronione. Kontrola dostępu do zasobów chronionych możliwa jest dzięki procesowi autoryzacji. Proces autoryzacji oparty jest o identyfikację tożsamości klienta aplikacji oraz uwierzytelnienie zidentyfikowanej tożsamości klienta. Rozwiązania dostarczane przez platformę Java EE skupiają się na aspektach bezpieczeństwa związanych z uwierzytelnianiem, autoryzacją dostępu do zasobów aplikacji, zapewnianiem integralności danych oraz poufności przekazywania informacji, logowaniem transakcji, audytami i zagadnieniami zapewniania jakości dostarczanych usług(ang. \textit{Quality of Service}). Mechanizmy bezpieczeństwa platformy Java EE dostarczane są przez kontener komponentów aplikacji. 

	\subsection{Poziomy mechanizmów bezpieczeństwa platformy Java EE}

		Mechanizmy bezpieczeństwa platformy Java EE dostarczane są na różnych poziomach - na poziomie warstwy aplikacji, warstwy transportowej lub na poziomie wiadomości.

		Mechanizmy bezpieczeństwa na poziomie aplikacji to usługi bezpieczeństwa dostarczane przez kontener aplikacji do wykorzystania przez konkretne aplikacje. Mechanizmy tego typu charakteryzują się łatwością konfiguracji i implementacji oraz pozwalają na dokładne dopasowanie pomiędzy mechanizmami kontroli dostępu do aplikacji a aplikacjami. Wadą rozwiązań tego typu jest brak możliwości zapewnienia bezpieczeństwa zasobów aplikacji poza jej środowiskiem. Ma to szczególnie duże znaczenie dla aplikacji i usług sieciowych. Dlatego mechanizmy bezpieczeństwa warstwy aplikacji powinny być uzupełnione mechanizmami warstwy transportowej i mechanizmami na poziomie przesyłanych wiadomości. 

		Zabezpieczenia warstwy transportowej dostarczane są przez mechanizmy przesyłania wiadomości pomiędzy dostawcami a klientami usług. Mechanizmy bezpieczeństwa warstwy transportowej platformy Java EE oparte są o szyfrowaną wersję protokołu HTTP - HTTPS wykorzystująca protokół SSL(\textit{Secure Sockets Layer}). Są to mechanizmy stosowane w połączeniach punkt - punkt pomiędzy nadawcą a odbiorcą. Umożliwiają uwierzytelnianie klienta i serwera, pozwalają gwarantować integralność i poufność przesyłanych informacji. Zawierają mechanizmy negocjacji algorytmów szyfrowania i kluczy kryptograficznych. Mechanizmy bezpieczeństwa warstwy transportowej gwarantują ochronę wiadomości na drodze pomiędzy nadawcą a odbiorcą. Rozwiązania oparte o protokół SSL wykorzystują certyfikaty cyfrowe w procesie uwierzytelniania uczestników komunikacji. 

		Zastosowanie mechanizmów bezpieczeństwa poziomu wiadomości polega na dołączeniu informacji bezpieczeństwa do przesyłanych wiadomości lub ich załączników. Informacja bezpieczeństwa może dotyczyć całości informacji lub jej części i jest z nią związana na całej drodze pomiędzy nadawcą a odbiorcą. Zastosowanie mechanizmów tego typu pozwala na bezpieczne przesyłanie informacji pomiędzy nadawcą a odbiorcą z pośrednictwem dowolnych węzłów komunikacyjnych. Ma na celu zagwarantowanie, że tylko odbiorca będzie w stanie odczytać zabezpieczone informacje.

	\subsection{Zastosowanie domen bezpieczeństwa w aplikacjach platformy Java EE}

		Mechanizmy bezpieczeństwa platformy Java EE oparte są o koncepcję domen bezpieczeństwa. W ramach domen bezpieczeństwa istnieją użytkownicy przypisani do różnych ról oraz klasyfikowani w ramach grup. Zastosowanie ról pozwala na przyznanie dowolnym klientom uprawnień do określonych zasobów aplikacji. Dzięki użyciu grup możliwe jest zbiorcze określenie uprawnień użytkowników przypisanych do grup. Określenie uprawnień do zasobów aplikacji możliwe jest między innymi dzięki użyciu adnotacji - \textit{@RolesAllowed}, \textit{@PermitAll}, \textit{@DenyAll}.
		
		\lstset{language=Java}
		\begin{lstlisting}
@RolesAllowed("RoleA")
public void forRoleA() {
	// Metoda dostępna dla użytkowników przypisanych do roli RoleA
}

@PermitAll
public void forAll(){
	// Metoda dostępna dla wszystkich użytkowników
}

@DenyAll
public void forbidden() {
	// Metoda niedostępna dla wszystkich użytkowników
}
		\end{lstlisting}		

%---------------------------------------------------------------------------

\autsection{Protokół LDAP}{Krzysztof Wilaszek}
\label{sec:ldap}

	LDAP(Lightweight Directory Access Protocol) jest protokołem definiującym metody, dzięki którym możliwy jest dostęp do danych zawartych w katalogach\cite{ZyTrax13}. Opisuje sposób reprezentacji danych w usłudze katalogowej oraz definiuje metody ładowania i eksportowania danych. Bazuje na standardzie X.500.

	\subsection{Model działania protokołu LDAP}

		\begin{figure}[h]
			\centering
			\includegraphics{img/ldap.jpg}
			\caption{Schemat funkcjonowania protokołu  LDAP}
			\label{Schemat funkcjonowania protokolu  LDAP}
		\end{figure}

		Protokół LDAP definiuje sposób dostępu do zasobów, nie określa natomiast sposób przechowywania danych. Jednym z wariantów jest przechowywanie informacji w bazie danych. Użytkownik komunikujący się z serwerem LDAP nie wie skąd pochodzą dane, które otrzymuje. Protokół LDAP definiuje również metody ładowania danych do usługi katalogowej oraz eksportowania danych z usługi przy użyciu formatu LDIF. Protokół LDAP opisuje operacje jakie mogą być wykonywane na modelu danych(np. modyfikowanie, usuwanie, odczyt).

	\subsection{Reprezentacja danych w protokole LDAP}

		Dane w protokole LDAP reprezentowane są w postaci hierarchii obiektów\cite{ZyTrax13}. Obiekty tworzą strukturę drzewa nazywanego drzewem DIT(Data Information Tree). Wierzchołek drzewa to element ,,root''. Każdy element drzewa składa się z co najmniej jednej klasy obiektów. Klasa obiektów to zbiór atrybutów przypisanych do obiektu. Każdy atrybut posiada nazwę i najczęściej ma przypisaną wartość. Klasa obiektu określa czy nadanie wartości danego atrybutu jest obowiązkowe czy opcjonalne. Obiekty identyfikowane są poprzez swoje położenie w drzewie - ścieżkę określającą elementy drzewa na drodze do obiektu. Identyfikator obiektu nazywany jest elementem Distinguished Name(DN).

	\subsection{Uwierzytelnianie przy użyciu protokołu LDAP}

		Usługa katalogowa LDAP może być wykorzystywana jako baza informacji o użytkownikach i serwis uwierzytelniający. LDAP pozwala również na przechowywanie informacji o rolach użytkowników. 

		Proces uwierzytelniania przy użyciu protokołu LDAP rozpoczyna się od pozyskania mapowania pomiędzy identyfikatorem użytkownika w systemie a elementem DN usługi katalogowej. Następnie wykonywana jest operacja uwierzytelniania w serwerze LDAP - ,,bind''. Aby wykonać operację ,,bind'' konieczne jest przesłanie nazwy DN i hasła użytkownika.  Innym sposobem uwierzytelnienia użytkownika może być porównanie przedstawionego przez niego hasła z hasłem przechowywanym przez usługę katalogową. Dla uwierzytelnionego użytkownika możliwe jest pobranie listy jego ról i uprawnień.

%---------------------------------------------------------------------------

\autsection{Mechanizmy bezpieczeństwa serwera aplikacyjnego JBoss}{Krzysztof Wilaszek}
\label{sec:jboss}

	Mechanizmy bezpieczeństwa serwera aplikacyjnego JBoss w wersji 7 oparte są o framework PicketBox. PicketBox dostarcza podstawowych funkcjonalności zapewnienia bezpieczeństwa dostępu do zasobów, takich jak uwierzytelnianie, autoryzacja, audyty systemu oraz mapowanie ról i danych uwierzytelniających. 

	\subsection{Podsystemy bezpieczeństwa serwera aplikacyjnego JBoss}

		Usługi bezpieczeństwa dostarczane przez Picketbox są dostępne dla serwera aplikacyjnego poprzez podsystem bezpieczeństwa(ang. Security Subsystem). Każdemu żądaniu klienta przypisywany jest kontekst bezpieczeństwa dostępny dla podsystemu bezpieczeństwa\cite{Lofthouse12}. 

		Kontekst bezpieczeństwa udostępnia komponenty skonfigurowane dla domeny bezpieczeństwa. Możliwe komponenty to:

		\begin{itemize}
			\item Authentication Manager - dokonuje uwierzytelniania użytkowników na podstawie otrzymanych danych uwierzytelniających przy użyciu modułów logowania zdefiniowanych dla wykorzystywanej domeny bezpieczeństwa;
			\item Authorization Manager - dostarcza informacji o rolach przypisanych użytkownikowi oraz dokonuje autoryzacji dostępu do zasobów dla uwierzytelnionych użytkowników;
			\item Audit Manager - pozwala na logowanie zdarzeń zachodzących w systemie zapewnienia bezpieczeństwa dostępu do aplikacji;
			\item Mapping Manager - pozwala na przypisywanie uwierzytelnionemu użytkownikowi dodatkowych uprawnień, ról lub atrybutów.
		\end{itemize}

		Korzystania z podsystemów bezpieczeństwa możliwe jest dzięki dodaniu rozszerzenia:
		\lstset{language=XML}
		\begin{lstlisting}
	<extension module="org.jboss.as.security"/>
		\end{lstlisting}
		w pliku konfiguracyjnym serwera.

		Podsystem bezpieczeństwa serwera aplikacyjnego JBoss udostępnia konfigurację następujących własności:

		\begin{itemize}
			\item security-management - pozwala nadpisywać domyślne parametry modułu PicketBox takie jak implementacje klas zarządców dla procesów uwierzytelniania, autoryzacji, audytów, mapowania danych tożsamości oraz tworzenia relacji zaufania.
			\item security-domains - pozwala na konfigurację domen bezpieczeństwa
			\item security-properties - pozwala definiować dodatkowe własności wymagane przez podsystem bezpieczeństwa.
		\end{itemize}

	\subsection{Domena bezpieczeństwa serwera aplikacyjnego JBoss}

		W ramach podsystemu bezpieczeństwa możliwa jest definicja domen bezpieczeństwa. Domena bezpieczeństwa opisuje mechanizmy zabezpieczeń dostępu do aplikacji wykorzystywane przez grupę usług przypisanych do tej domeny.

		Podstawowym zadaniem domeny bezpieczeństwa jest przeprowadzanie procesu uwierzytelniania klientów aplikacji. W tym celu do domeny przypisane są moduły logowania(ang. ,,Login Module'') wykorzystywane do uwierzytelniania użytkowników. Konfigurując moduł logowania należy wybrać klasę definiującą sposób i przebieg uwierzytelniania użytkownika. Domyślnie dostępne są implementacje pozwalające na uwierzytelnianie np. w oparciu o certyfikaty, bazę danych z użytkownikami i hasłami, protokół LDAP, protokół Kerberos lub prosty plik z użytkownikami i hasłami.

		Możliwe jest również oparcie mechanizmów uwierzytelniania o specyfikację JASPI(Java Authentication Service Provider Interface for Containers). JASPI definiuje standardowy interfejs dla dostawców usług, przy pomocy którego dla kontenera aplikacji Java EE możliwe jest uwierzytelnianie na podstawie danych bezpieczeństwa przesyłanych na poziomie wiadomości. Specyfikacja określa mechanizmy strony klienckiej oraz serwerowej. Serwer ma możliwość weryfikacji tokenów bezpieczeństwa lub podpisów przychodzących wiadomości i pozyskania opisu uprawnień użytkownika lub asercji. Strona kliencka może dodawać do wysyłanych wiadomości token bezpieczeństwa lub podpis cyfrowy. 

		Inny ważnym mechanizmem definiowanym na poziomie domeny bezpieczeństwa jest autoryzacja klientów aplikacji. Domyślnie realizowanym podejściem w procesie autoryzacji jest RBAC(Role Based Access Control). Metoda ta przydziela lub odmawia prawa dostępu do zasobów w oparciu o przynależność użytkownika do określonej grupy. Możliwe jest również użycie innych metod autoryzacji, np. JACC(Java Authorization Contract for Containers) lub XACML (eXtensible Access Control Markup Language). 

		Definicja domeny bezpieczeństwa obejmuje również:

		\begin{itemize}
			\item mapowania ról, uprawnień, danych uwierzytelniających i atrybutów; 
			\item konfigurację mechanizmu audytów operacji w domenie bezpieczeństwa
			\item konfigurację repozytorium certyfikatów bezpieczeństwa wykorzystywanych przez kontekst SSL lub przez procesy pozyskiwania i magazynowania certyfikatów.
		\end{itemize}

%---------------------------------------------------------------------------

\autsection{Picketlink}{Krzysztof Wilaszek}
\label{sec:picketlink}

Picketlink jest projektem dostarczającym implementacji założeń systemów zarządzania tożsamościami dla aplikacji Java. Picketlink definiuje model tożsamości wykorzystywanych w operacjach systemów IdM. Pozwala na zarządzanie informacjami o użytkownikach, grupach i rolach oraz ich atrybutach. Umożliwia wykorzystanie różnych sposobów przechowywania tożsamości, np. bazy danych lub usługi katalogowej LDAP. Projekt Picketlink umożliwia korzystanie z federacji tożsamości - wspiera specyfikacje SAML, OpenID oraz WS-Trust\cite{PicketLink13}.

Picketlink dostarcza wsparcie dla profili specyfikacji SAML umożliwiających jednokrotne uwierzytelnianie aplikacji webowych oraz globalne wylogowanie. Określa również mapowania dla protokołu HTTP przy użyciu komunikatów POST i Redirect. 

Picketlink implementuje mechanizmy pozwalające na tworzenie usługi ,,Identity Provider''. Dostarcza narzędzi konfiguracji usługi, uwierzytelniania a także implementację operacji obsługi przechowywania tożsamości. W ramach konfiguracji możliwe jest między innymi określenie zaufanych hostów dla usługi IdP, ustawienie cyfrowego podpisu dla wiadomości oraz szyfrowanie komunikatów. Usługa ,,Identity Provider'' w procesie uwierzytelniania może korzystać z mechanizmu domen bezpieczeństwa serwera aplikacyjnego JBoss. Dostawcy usług korzystają z serwisu IdP w pozyskując asercje opisujące użytkowników. Picketlink umożliwia taką konfigurację usług aby ich mechanizmy uwierzytelniania korzystały z usługi IdP.

Narzędzia Picketlink dostarczają implementacji specyfikacji WS-Trust - za operacje zarządzania asercjami SAML(tworzenie, odnawianie, anulowanie i walidację) odpowiada usługa ,,Security Token Service''. Moduł Picketlink zawiera implementację usługi STS oraz pozwala konfigurować parametry takie jak szyfrowanie tokenu bezpieczeństwa i wymóg dołączania cyfrowego podpisu dla tokenów. Moduł implementuje narzędzia przechwytywania wywołań usług webowych i weryfikacji wymaganych asercji uprawniających do dostępu do zasobów. 

%---------------------------------------------------------------------------

\section{Business Process Management}
\label{sec:bpm}

Business Process Management

%---------------------------------------------------------------------------

\section{Enterprise Service Bus}
\label{sec:esb}

Enterprise Service Bus

%---------------------------------------------------------------------------

\section{OpenShift}
\label{sec:openShift}

OpenShift jest środowiskiem typu Platform as a Service rozwijanym przez firmę Red Hat. Tworząc warstwę abstrakcji nad infrastrukturą z wykorzystaniem oprogramowania typu middleware i udostępniając zestaw odpowiednich narzędzi, platforma ta ułatwia tworzenie, wdrażanie i zarządzanie aplikacjami w wielu językach programowania. 

Środowisko OpenShift jest dostępne dla użytkowników w co najmniej dwóch postaciach. OpenShift Online to komercyjna usługa sieciowa, udostępniająca tworzenie aplikacji w chmurze publicznej. OpenShift Origin to otwarta wersja środowiska, która może być uruchomiona w dowolnej lokacji i w całości zarządzana przez użytkownika.
OpenShift jest zbudowany w oparciu o system Red Hat Enterprise Linux i może zostać uruchomiony wszędzie tam gdzie system ten działa, a więc także na np. maszynach wirtualnych. Architektura środowiska wyróżnia dwa rodzaje instancji systemu RHEL - node służy do uruchamiania aplikacji użytkowników, zaś broker ma za zadanie zarządzanie i orkiestrację instancji typu node. W ramach pojedynczej instancji node jest dostępnych wiele komponentów gear, które stanowią środowisko użytkownika końcowego, z zasobami ograniczonymi przy użyciu mechanizmów jądra systemu - Control Groups. Poszczególne komponenty gear są izolowane od siebie(multi-tenancy) za pomocą modułu SELinux. 

Z punktu widzenia użytkownika końcowego zasoby udostępniane przez OpenShift są zgrupowane w dwóch rodzajach komponentów.
Wspomniany wcześniej gear to komponent udostępniający fizyczne zasoby systemu takie jak przestrzeń dyskowa, zasoby procesora, pamięć operacyjna, łączność sieciowa. OpenShift Online udostępnia obecnie trzy wersje tego komponentu, różniące się rozmiarem dostępnej przestrzeni dyskowej i pamięci operacyjnej.
Cartridge jest komponentem zapewniającym możliwość uruchomienia aplikacji danego typu. Istnieją dwie klasy komponentu cartrige: primary oraz embedded. 
Cartridge typu primary kontrolują cykl życia aplikacji i udostępniają interfejs dostępny dla użytkownika tworzonej aplikacji, np. w postaci aplikacji internetowej lub usługi sieciowej. Zazwyczaj umożliwiają one budowę oraz uruchomienie aplikacji napisanej w konkretnym języku programowania, dostarczając interpretery i środowiska uruchomieniowe oraz serwer aplikacji. OpenShift Online dostarcza gotowych komponentów, pozwalających na tworzenie aplikacji w językach Java, Ruby, Python, PHP, Perl, Go, a także z wykorzystaniem popularnych frameworków dla tych języków takich jak Django czy Ruby on Rails. Maksymalnie jeden cartridge typu primary może być uruchomiony w danym komponencie gear. 
Cartridge embedded mogą korzystać z osobnego komponentu gear ale mogą również być współdzielić go z cartridgem typu primary. Zazwyczaj dostarczają one funkcji które nie są bezpośrednio dostępne dla końcowego użytkownika aplikacji, a są jedynie wykorzystywane przez aplikacje uruchomione w cartridge’ach primary. Przykłady tego komponentów tego typu to bazy danych MySQL, PostgreSQL, MongoDB czy framework SwitchYard.
W przypadku gdy język bądź aplikacja nie jest wspierana przez OpenShift, użytkownik platformy może utworzyć nowy typ cartridge’u lub wykorzystać cartridge DIY(Do-It-Yourself), który pozwala na uruchomienie aplikacji zgodnej z systemem Linux. W obu przypadkach trzeba jednak mieć świadomość znacznych ograniczeń narzucanych przez system dla zachowania izolacji pomiędzy aplikacjami różnych użytkowników.


%---------------------------------------------------------------------------
\chapter{Opis implementacji}
\label{cha:implementacja}

Opisy poszczególnych komponentów systemu, z wyszczególnieniem ich przeznaczenia, intefejsów i zależności od innych komponentów.

%---------------------------------------------------------------------------

\chapter{Analiza zastosowań tworzonego systemu}
\label{cha:zastosowania}

{\it

Niniejszy rozdział przedstawia możliwości zastosowania proponowanych rozwiązań w rzeczywistych systemach o architekturze SOA oraz wynikające z tego korzyści. Wykorzystanie mechanizmów systemów zarządzania tożsamościami pozwala zapewnić bezpieczeństwo aplikacji o architekturze zorientowanej na usługi. Zastosowany standard SAML i definiowane przy jego użyciu mechanizmy stanowią rozwiązanie dobrze dopasowane do specyfiki systemów o architekturze SOA. Elastyczność i skalowalność proponowanego podejścia sprawiają, że możliwe jest jego efektywne wdrożenie w rzeczywistym środowisku systemu bazującego na architekturze SOA.

Rozdział opisuje również potencjalne ulepszenia proponowanego rozwiązania. Zwiększenie poziomu bezpieczeństwa i rozszerzenie funkcjonalności systemu mogłoby być możliwe np. dzięki: wprowadzeniu wielu usług uwierzytelniających STS powiązanych relacją zaufania, zastosowaniu szyfrowania na poziomie fragmentów przesyłanych wiadomości, możliwości definiowania parametrów bezpieczeństwa przy użyciu polityk lub integracji usług STS(używanych przez serwisy webowe) z usługami IdP(używanymi przez aplikacje przeglądarkowe).

}

%---------------------------------------------------------------------------

\section{Możliwości wykorzystania środowiska w praktyce}
\label{sec:wykorzystanieWPraktyce}

	System powstały w wyniku wykonanych prac projektowych stanowi przykład wykorzystania podejścia opartego o zarządzanie tożsamościami jako metody zapewnienia bezpieczeństwa dostępu do aplikacji w architekturze SOA. Przygotowana implementacja dowodzi, że  istnieje możliwość oparcia funkcjonowania mechanizmów bezpieczeństwa aplikacji w architekturze SOA na rozwiązaniach dostarczanych przez systemy zarządzania tożsamościami. 

	Wynikiem prac implementacyjnych jest system wykorzystujący standard SAML jako realizację założeń systemów zarządzania tożsamościami. Przygotowany system dostarcza przykładów zastosowania specyfikacji SAML jako metody uwierzytelniania klientów usług sieciowych. Wykorzystanie  SAML w procesie uwierzytelniania umożliwia implementację mechanizmów bezpieczeństwa dla usług opartych o różne standardy dostarczania usług sieciowych(np. SOAP i REST). Do architektury przykładowego systemu wprowadzono również moduł magistrali usług - ESB, będący charakterystycznym elementem systemów opartych o architekturę SOA. Zaproponowana implementacja modułu magistrali usług dostarcza funkcjonalności przetwarzania wiadomości z dołączonymi tokenami bezpieczeństwa - asercjami języka SAML. Dzięki zastosowaniu mechanizmów bezpieczeństwa bazujących na specyfikacji SAML możliwe było osiągnięcie funkcjonalności jednokrotnego uwierzytelniania klienta dla różnych usług systemu o architekturze typu SOA. Mechanizm jednokrotnego uwierzytelniania może być wykorzystany podczas budowania procesu biznesowego odwołującego się do różnych usług sieciowych.

	Zastosowanie mechanizmów systemów zarządzania tożsamościami w kontekście aplikacji w architekturze SOA jest rozwiązaniem posiadającym wiele zalet. Może stanowić sposób standaryzacji implementacji mechanizmów zapewniania bezpieczeństwa aplikacji. Opiera się na oddelegowaniu odpowiedzialności związanych z bezpieczeństwem aplikacji do specjalizowanych usług. Dzięki zastosowaniu standardu SAML możliwe jest tworzenie relacji zaufania pomiędzy różnymi usługami uwierzytelniania i autoryzacji. Dzięki temu klient aplikacji może w transparentny dla siebie sposób korzystać z usług w obrębie różnych domen bezpieczeństwa. Zastosowanie zarządzania tożsamościami jest podejściem elastycznym, skalowalnym, umożliwiającym efektywne rozszerzanie na kolejne elementy infrastruktury systemu. Oddelegowanie odpowiedzialności związanych z uwierzytelnianiem i autoryzacją do odrębnej, specjalizowanej usługi zwiększa poziom bezpieczeństwa systemu. Wykorzystanie specyfikacji SAML umożliwia realizację procesu jednokrotnego uwierzytelniania klienta korzystającego z wielu usług sieciowych.

	Proces zapewnienia bezpieczeństwa dostępu do systemu opartego o architekturę SOA może być realizowany z wykorzystaniem mechanizmów zarządzania tożsamościami. Zaproponowane rozwiązanie wykorzystujące standard SAML jest podejściem dobrze dopasowanym do specyfiki i wymagań systemów o architekturze SOA.

%---------------------------------------------------------------------------

\section{Użyteczność zastosowanych technologii w systemach udostępniania usług}
\label{sec:uzytecznosc}

Użyteczność zastosowanych technologii w systemach udostępniania usług

%---------------------------------------------------------------------------

\section{Proponowane ulepszenia}
\label{sec:ulepszenia}

Proponowane ulepszenia

%---------------------------------------------------------------------------

\chapter{Wnioski}
\label{cha:wnioski}

Niniejsza praca przedstawia podejście do problemu zapewnienia bezpieczeństwa aplikacji w architekturze zorientowanej na usługi przy użyciu koncepcji systemów zarządzania tożsamościami. W ramach związanych z nią prac badawczych zaprojektowano i zaimplementowano system stanowiący przykład wykorzystania mechanizmów zarządzania tożsamościami dla aplikacji w architekturze SOA. Jako standard realizujący założenia systemów zarządzania tożsamościami proponowany system wykorzystuje specyfikację SAML. 

Przykładowy system prezentowany w pracy dowodzi użyteczności zastosowania koncepcji systemów typu \textit{IdM} w kontekście aplikacji opartych o architekturę SOA. Wykorzystanie asercji SAML umożliwiło implementację mechanizmów uwierzytelniania klientów serwisów webowych, dostarczanych przy użyciu różnych standardów udostępniania usług sieciowych(SOAP, REST). Zastosowanie protokołu SAML pozwoliło na implementację mechanizmów bezpieczeństwa systemu wykorzystującego różnorodne usługi w celu dostarczania wymaganych funkcjonalności. 

Wykorzystanie tokenów bezpieczeństwa bazujących na asercjach SAML jest rozwiązaniem dobrze dopasowanym do specyfiki aplikacji w architekturze SOA oraz do potrzeb związanych z wdrażaniem aplikacji na platformach chmur obliczeniowych. Jest podejściem realizującym założenia mechanizmów bezpieczeństwa na poziomie przesyłanych wiadomości. Dzięki temu pozwala na stosunkowo prostą implementację elementów charakterystycznych dla architektury typu SOA, w tym magistrali usług(ESB). Jest rozwiązaniem elastycznym; umożliwiającym efektywne wdrożenie dla różnorodnych dostawców i odbiorców usług oraz elementów pośrednich. Implementacja mechanizmów bezpieczeństwa w oparciu o protokół SAML umożliwia realizację procedury jednokrotnego uwierzytelniania klienta procesu biznesowego wykorzystującego niezależne od siebie usługi.

Zastosowanie mechanizmów bezpieczeństwa opartych o koncepcję zarządzania tożsamościami upraszcza rozwiązania części problemów, obserwowanych dla innych rozwiązań zapewniania bezpieczeństwa aplikacji w architekturze SOA. Jest podejściem bardziej efektywnym w procesie wdrażania dla poszczególnych elementów w rozbudowanej infrastrukturze systemu o architekturze SOA. Jest rozwiązaniem skalowalnym; charakteryzującym się łatwością rozszerzania - również poza granice domen wyznaczonych przez zasięg poszczególnych usług uwierzytelniających.

Przedstawione rozwiązanie nie wykorzystuje wszystkich korzyści wiążących się z zastosowaną specyfikacją SAML jak i z innymi wykorzystanymi standardami, bibliotekami i narzędziami. Dlatego istnieje kilka potencjalnych ulepszeń systemu oraz tematów dalszych badań. Jednym z możliwych ulepszeń byłoby zastosowanie szyfrowania na poziomie fragmentów przesyłanych wiadomości. W przypadku serwisów przesyłających dane w formacie XML możliwe byłoby zastosowanie standardu XML Encrytpion. Brak jednak odpowiednich standardów dla informacji przesyłanych w formacie JSON. Innym potencjalnym ulepszeniem byłoby definiowanie możliwości i wymagań aplikacji w kontekście bezpieczeństwa przy użyciu mechanizmu polityk. Ważnym udoskonaleniem działania systemu byłoby wprowadzenie dodatkowych usług uwierzytelniających - STS i powiązanie ich relacją zaufania. Dzięki temu możliwe byłoby jednokrotne uwierzytelnianie klienta dla obsługi przez serwisy używające odrębnych usług bezpieczeństwa. Kolejne korzyści mogłaby przynieść integracja usługi STS odpowiedzialnej za dostarczanie i walidację tokenów bezpieczeństwa na potrzeby uwierzytelniania klientów serwisów z usługą IdP wykorzystywaną w procesie uwierzytelnia użytkowników aplikacji webowych.

Przedstawione podejście do problemu zapewnienia bezpieczeństwa systemów o architekturze SOA mogłoby zostać wykorzystane w rzeczywistych, złożonych systemach. Zaproponowane rozwiązanie jest dobrze dopasowane do charakteru aplikacji o architekturze typu SOA. Zastosowanie prezentowanego modelu pozwala na realizację procesu jednokrotnego uwierzytelniania. Pozwala na transparentne przekraczanie granic domen bezpieczeństwa. Możliwość wielokrotnego użycia rozwiązania dla różnych typów usług i jego skalowalność pozwalają na jego efektywne wdrożenie w rozbudowanej infrastrukturze systemu o architekturze SOA. Zastosowanie proponowanego podejścia pozwala zwiększyć poziom bezpieczeństwa systemu oraz rozszerzyć możliwości związane z uwierzytelnianiem i autoryzacją dostępu do jego zasobów.

%---------------------------------------------------------------------------



% itd.
% \appendix
% \include{dodatekA}
% \include{dodatekB}
% itd.

\bibliographystyle{alpha}
\bibliography{bibliografia}
%\begin{thebibliography}{1}
%
%\bibitem{Dil00}
%A.~Diller.
%\newblock {\em LaTeX wiersz po wierszu}.
%\newblock Wydawnictwo Helion, Gliwice, 2000.
%
%\bibitem{Lam92}
%L.~Lamport.
%\newblock {\em LaTeX system przygotowywania dokumentów}.
%\newblock Wydawnictwo Ariel, Krakow, 1992.
%
%\bibitem{Alvis2011}
%M.~Szpyrka.
%\newblock {\em {On Line Alvis Manual}}.
%\newblock AGH University of Science and Technology, 2011.cccccc
%\newblock \\\texttt{http://fm.ia.agh.edu.pl/alvis:manual}.
%
%\end{thebibliography}

\end{document}
